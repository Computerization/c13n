\title{Git 变基(Rebase)基础教程}
\author{叶家炜}
\date{Jan 13, 2026}
\maketitle
Git 变基是一种强大的工具,它将一个分支的提交「重新应用」到另一个分支上,从而实现干净的线性历史记录。这种操作不同于传统的合并,它能避免多余的合并提交,让项目历史看起来更加简洁明了。变基的核心在于重新排列提交序列,使分支间的关系更直观。\par
相比于 merge 操作,变基的主要优势在于保持历史线性。Merge 会创建一个额外的合并提交,记录两个分支的融合过程,这在多人协作时可能导致历史记录变得杂乱。而变基则将 feature 分支的变更「移植」到主分支顶端,形成一条平滑的直线。这种方式特别适合个人开发或清理提交历史,但需要注意它会改写提交历史,因此不适用于已共享的分支。\par
变基与 merge 的直观对比可以想象为:merge 像两条河流汇合形成一个分叉口,而变基则像将一条支流顺直地接续到主河道上。前者保留了所有历史痕迹,后者追求简洁的单一线性路径。这种差异在长期项目中尤为明显,线性历史更容易 bisect 查找问题。\par
本文面向 Git 新手到中级用户,如果你已经掌握基本的 commit、branch 和 checkout 命令,就可以轻松跟进。阅读前提包括理解 Git 的基础工作流,如创建分支和切换分支。没有这些基础,建议先复习 Git 官方入门文档。\par
文章结构从基础概念入手,逐步深入到实际操作、高级技巧、问题排查和最佳实践,最后提供快速参考和练习建议。通过层层递进,你将掌握变基的全貌。\par
\chapter{变基基础概念}
变基的工作原理本质上是将当前分支的提交从其原有基点「剥离」,然后逐一重新应用到目标分支的顶端。具体过程是:Git 首先找到两个分支的共同祖先提交,然后将当前分支独有的提交「暂停」,切换到目标分支,再按顺序「replay」这些提交。每个 replay 过程相当于 cherry-pick 一个提交,如果有冲突则暂停等待用户解决。这种「移植」机制确保了提交内容的完整性,但会生成全新的提交哈希值。\par
常见变基场景包括当前分支变基到目标分支,使用命令 \texttt{git rebase <target>},这会将当前分支的变更叠加到 target 分支上。另一个场景是交互式变基,通过 \texttt{git rebase -i} 可以编辑提交序列,比如合并或删除提交。这两种场景覆盖了 90\%{} 的使用需求。\par
变基过程中有三种状态:正在进行时,Git 会标记 rebase 状态文件;已暂停状态通常因冲突发生,需要手动干预;已完成状态则一切顺畅,历史已重写。理解这些状态有助于诊断问题。\par
关键术语中,Base Commit 是变基的基准提交,即目标分支的顶端;Replay 表示重新应用提交的过程;Pick 是交互式变基中的默认动作,意为保持原样;Squash 则将当前提交合并到上一个提交中,结合它们的变更和日志。\par
\chapter{环境准备}
要开始学习变基,首先创建示例仓库。执行以下命令序列:\texttt{git init rebase-demo},然后 \texttt{cd rebase-demo}。接下来创建初始提交,例如触碰一个文件 \texttt{echo "Initial commit" > README.md} 并 \texttt{git add .},最后 \texttt{git commit -m "Initial commit"}。这个仓库将作为所有演示的基础。\par
在仓库中创建测试分支结构:在 main 分支上添加几个提交,如 \texttt{echo "Main change 1" >> README.md}、\texttt{git add .}、\texttt{git commit -m "Main change 1"},重复几次。然后创建 feature 分支 \texttt{git checkout -b feature},并在其上添加独有提交,如 \texttt{echo "Feature change" >> README.md}、\texttt{git commit -m "Feature change"}。现在你有 main 和 feature 两条平行分支,完美模拟真实开发场景。\par
为提升体验,配置 \texttt{git config --global rebase.autoStash true},这会在变基时自动暂存未提交变更,避免手动 stash。推荐工具包括 Git GUI 用于可视化历史,以及 VS Code 的 Git Graph 扩展来观察分支变化。\par
\chapter{基本变基操作}
\section{简单变基}
简单变基是最基础的操作,假设你在 feature 分支上,执行 \texttt{git checkout feature},然后 \texttt{git rebase main}。这个命令的解读如下:首先切换到 feature 分支,确保它是干净的;然后 rebase main 告诉 Git 将 feature 的提交从 main 的顶端重新应用。Git 会找到 main 和 feature 的分叉点,将 feature 之后的提交逐一 replay 到 main 顶端。如果无冲突,feature 分支现在「骑」在 main 上,形成线性历史。\par
预期结果是:变基前,main 和 feature 平行;变基后,feature 的提交直接接在 main 末尾,原有 feature 基点被遗弃。新提交有全新哈希,但内容相同。这种操作常用于将本地 feature 同步到远程 main 前,保持干净历史。\par
\section{处理变基冲突}
变基冲突发生在 replay 提交时,变更与目标分支重叠。机制是 Git 尝试应用补丁,如果文件行冲突则标记 <<<<<<< 等符号。解决步骤:先 \texttt{git status},它会显示「rebase in progress」和冲突文件;编辑冲突文件,手动选择保留哪部分代码;然后 \texttt{git add .} 标记已解决;最后 \texttt{git rebase --continue} 继续下一个提交。\par
完整示例:假设 main 有 \texttt{echo "foo" > file.txt},feature 有 \texttt{echo "bar" >> file.txt},变基时冲突。编辑后文件可能成 \texttt{foo\textbackslash{}nbar},add 并 continue。整个过程确保变更不丢失,但需仔细审查。\par
\section{中止变基}
如果冲突太棘手,使用 \texttt{git rebase --abort}。这个命令解读为:中止当前变基,恢复到 rebase 开始前的分支状态,包括 HEAD 和索引。它会删除 rebase 状态文件,一切如初。使用时机是当你不确定如何解决冲突,或变基策略错误时;必须使用则是如果误操作导致不可逆混乱。\par
\chapter{交互式变基}
\section{基本语法}
交互式变基通过 \texttt{git rebase -i HEAD\~{}3} 编辑最近 3 个提交。这个命令解读:\texttt{-i} 启用交互模式,HEAD\~{}3 指定从倒数第三个提交开始的范围。Git 会弹出编辑器,显示提交列表,默认全为 pick。保存退出后,Git 按指令执行。\par
另一种是 \texttt{git rebase -i main},将当前分支变基到 main 前,同时交互编辑。这适合将 feature 的提交精简后叠加到 main。\par
\section{常用操作命令详解}
交互式变基的核心是编辑器中的命令。\texttt{pick} 保持提交不变,是默认选项,用于正常保留。\texttt{reword} 只修改提交信息,如修正拼写,Git 会暂停让你编辑消息后继续。\texttt{edit} 在该提交处暂停,允许修改代码或作者,然后 \texttt{git rebase --continue}。\par
\texttt{squash} 将当前提交合并到上一个,结合变更并让你编辑合并消息,常用于清理小修复。\texttt{fixup} 类似 squash 但丢弃当前提交信息,直接融入上一个,适合临时提交。\texttt{drop} 完全删除提交,用于移除错误。\par
\section{实战案例}
修改最近提交信息:\texttt{git rebase -i HEAD\~{}1},将 pick 改为 reword,保存后编辑消息如从「Fix bug」改为「修复登录验证 bug」,继续即可。\par
合并多个小提交:\texttt{git rebase -i HEAD\~{}3},将后两个改为 squash,编辑器出现合并消息界面,合成「feat: 添加用户模块」。\par
删除错误提交:\texttt{git rebase -i HEAD\~{}4},将目标行改为 drop,保存后该提交消失。\par
分离大提交:先 \texttt{git reset HEAD\~{}1},然后重新 commit 分拆,最后 \texttt{git rebase -i HEAD\~{}n} 调整顺序。\par
\chapter{高级变基技巧}
\section{变基到上游分支}
\texttt{git rebase --onto main featureA featureB} 将 featureB 从 featureA 之后的提交变基到 main 上。解读:\texttt{--onto main} 指定新基点,featureA 是旧基点分界,featureB 是目标分支。这常用于将变更从一个分支「移植」到另一个上游,常在多分支协作中应用。\par
\section{保留特定提交}
\texttt{git rebase --onto new-base old-base} 将当前分支从 old-base 之后的提交应用到 new-base。解读:old-base 是保留前缀的分界,新提交只 replay old-base 之后部分。这用于精确控制历史片段。\par
\section{批量修改提交作者}
\texttt{git rebase -i --exec "git log --oneline -1" HEAD\~{}5} 在每个 pick 后执行命令。解读:\texttt{-i} 交互,\texttt{--exec} 指定每次暂停运行 \texttt{git log --oneline -1} 查看最新提交,HEAD\~{}5 范围为最近 5 个。实际中可换成 \texttt{git commit --amend --author="New Author"} 批量改作者。\par
\section{变基公共分支的最佳实践}
绝对不要对已推送公共分支变基,因为它改写历史会导致他人拉取混乱。如果必须推送,使用 \texttt{git push --force-with-lease},它检查远程是否变化,安全覆盖。\par
\chapter{常见问题与解决方案}
遇到「Cannot rebase: already in progress」是因为变基未完成,使用 \texttt{git rebase --abort} 清理或 \texttt{--continue} 推进。冲突解决后提交丢失可能是未正确 continue,检查 \texttt{git reflog} 找到旧 HEAD 并 reset 恢复。变基后历史混乱通常是对公共分支操作,重置 \texttt{git reset --hard origin/main}。交互式保存失败源于编辑器,配置 \texttt{git config --global core.editor "code --wait"} 解决。\par
\chapter{最佳实践与注意事项}
变基推荐用于个人分支和清理历史,如 feature 分支变基前推 main。禁止用于已推送公共分支或共享分支,以免团队冲突。在团队中,策略是 feature 变基到 main 后 merge。变基前检查清单:确认分支干净、无未推提交、备份 reflog。\par
与 Git Flow 结合,在 release 前变基 feature;GitHub Flow 中,PR 前变基保持线性。\par
\chapter{快速参考命令表}
基础变基:\texttt{git rebase main} 将当前变基到 main;\texttt{git rebase --abort} 中止;\texttt{git rebase --continue} 继续。\par
交互式:\texttt{git rebase -i HEAD\~{}n} 编辑最近 n 个;\texttt{git rebase -i --autosquash} 自动处理 fixup。\par
高级:\texttt{git rebase --onto A B C} 将 C 从 B 到 A;\texttt{git push --force-with-lease} 安全推送。\par
\chapter{实践练习}
练习 1:基础变基,在示例仓库 \texttt{git checkout feature}、\texttt{git rebase main},观察 \texttt{git log --oneline --graph}。\par
练习 2:交互式合并,在 feature 添加 3 小提交,\texttt{git rebase -i HEAD\~{}3} squash 后两个。\par
练习 3:制造冲突,编辑相同行后解决并 continue。\par
练习 4:故意出错,用 \texttt{git reflog} 恢复。完整仓库可在 GitHub rebase-demo 下载实践。\par
\chapter{结论}
变基关键要点:线性历史、交互编辑、冲突处理、安全推送。它的价值在于干净历史促进高效协作。下步学习 Git LFS 或 Submodules。鼓励立即实践,形成肌肉记忆。\par
\chapter{附录}
图形工具如 GitKraken 可视化变基。官方文档:\texttt{git rebase --help}。常见错误:NO-REBASE-OPTION 用 abort;FAQ 示例:变基是否改哈希?是,新提交全新 ID。\par
