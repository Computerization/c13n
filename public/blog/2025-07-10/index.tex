\title{"循环链表"}
\author{"黄京"}
\date{"Jul 10, 2025"}
\maketitle
在数据结构领域,单链表是一种基础且广泛使用的线性结构。然而,单链表存在一个显著局限性:尾节点操作效率低下。例如,在单链表中插入或删除尾节点时,必须从头节点开始遍历整个链表,时间复杂度为 $O(n)$,其中 $n$ 为节点数量。这种效率问题在需要频繁操作尾部的场景中尤为突出。循环链表的核心理念正是通过构建闭环结构来解决这一边界问题。其本质是将尾节点的指针指向头节点,形成一个无始无终的环。这种设计消除了单链表的“终点”概念,使得头尾操作变得高效。典型应用场景包括操作系统进程调度中的轮询算法、游戏开发中的角色循环队列,以及音频流处理中的数据缓冲区。在这些场景中,循环链表的环形特性天然支持连续遍历和高效拼接。\par
\chapter{循环链表基础解析}
循环链表的核心在于其闭环结构。在单向循环链表中,尾节点的 \texttt{next} 指针指向头节点;而双向循环链表则增加了 \texttt{prev} 指针,实现双向闭环。关键特性是空链表的表示方式:当链表为空时,头指针满足 \texttt{head->next = head}。这与单链表使用 \texttt{NULL} 表示空节点形成本质区别。遍历循环链表时,终止条件不再是 \texttt{current != NULL},而是 \texttt{current != head}。这意味着遍历从任意节点开始,最终会返回起点。插入或删除头节点时,指针维护逻辑也不同于单链表。例如,删除头节点需修改尾节点的指针以维持闭环,否则会导致结构断裂。\par
\chapter{循环链表的操作实现(附 C 代码)}
实现循环链表的第一步是定义节点结构。以下代码展示了节点定义和初始化函数:\par
\begin{lstlisting}[language=c]
typedef struct Node {
    int data;               // 数据域,存储整数值
    struct Node* next;      // 指针域,指向下一个节点
} Node;

Node* create_node(int data) {
    Node* new_node = (Node*)malloc(sizeof(Node));  // 动态分配内存
    new_node->data = data;                        // 设置数据值
    new_node->next = new_node;                    // 初始化自环,确保新节点指向自身
    return new_node;                              // 返回新节点指针
}
\end{lstlisting}
这段代码创建了一个新节点,并通过 \texttt{new\_{}node->next = new\_{}node} 实现自环初始化。这是循环链表的基础,确保单个节点也能形成闭环。\par
核心操作包括插入、删除和遍历。在空链表插入时,直接将头指针指向新节点:\texttt{head = new\_{}node;}。头插法操作如下:\par
\begin{lstlisting}[language=c]
new_node->next = head->next;  // 新节点指向原头节点的下一个节点
head->next = new_node;        // 头节点指向新节点,完成插入
\end{lstlisting}
此操作在 $O(1)$ 时间内完成。尾插法则需定位尾节点:\par
\begin{lstlisting}[language=c]
Node* tail = head;
while (tail->next != head) {  // 遍历至尾节点
    tail = tail->next;
}
tail->next = new_node;       // 尾节点指向新节点
new_node->next = head;       // 新节点指向头节点,维持闭环
\end{lstlisting}
尾插法的时间复杂度为 $O(n)$,但通过维护尾指针可优化至 $O(1)$。\par
删除操作需特别注意边界处理。删除头节点示例:\par
\begin{lstlisting}[language=c]
if (head->next == head) {    // 单节点情况
    free(head);
    head = NULL;
} else {
    Node* prev_tail = head;
    while (prev_tail->next != head) {  // 定位头节点的前驱(尾节点)
        prev_tail = prev_tail->next;
    }
    prev_tail->next = head->next;  // 尾节点指向新头节点
    free(head);                   // 释放原头节点
    head = prev_tail->next;       // 更新头指针
}
\end{lstlisting}
删除中间节点时,逻辑与单链表类似,但需额外维护闭环。\par
遍历循环链表使用 \texttt{do-while} 循环确保至少执行一次:\par
\begin{lstlisting}[language=c]
void print_list(Node* head) {
    if (!head) return;           // 空链表直接返回
    Node* current = head;
    do {
        printf("%d ", current->data);  // 打印当前节点数据
        current = current->next;       // 移至下一节点
    } while (current != head);         // 终止条件:返回头节点
}
\end{lstlisting}
⚠️ 关键陷阱:若误用 \texttt{while (current != NULL)} 会导致死循环,因为循环链表无 \texttt{NULL} 指针。\par
特殊边界处理包括单节点删除(直接释放内存并置空头指针)和约瑟夫环问题中的删除模式。后者涉及周期性删除节点,需精确控制遍历步长。\par
\chapter{循环链表的优势与代价}
循环链表的优势显著。头尾拼接操作在 $O(1)$ 时间内完成,优于单链表的 $O(n)$。例如,拼接两个循环链表只需修改尾节点指针。环形遍历无需边界判断,简化了迭代逻辑。在实现旋转缓冲区(如音频流)或轮询系统时,循环链表是天然选择。下表对比了关键操作的时间复杂度:\par
\begin{table}[H]
\centering
\begin{tabular}{|l|l|l|}
\hline
操作 & 单链表时间复杂度 & 循环链表时间复杂度 \\
\hline
头插法 & $O(1)$ & $O(1)$ \\
\hline
尾插法 & $O(n)$ & $O(n)$(可优化至 $O(1)$) \\
\hline
头尾拼接 & $O(n)$ & $O(1)$ \\
\hline
遍历 & $O(n)$ & $O(n)$ \\
\hline
\end{tabular}
\end{table}
然而,循环链表也存在缺陷。⚠️ 内存泄漏风险较高:循环引用需手动释放所有节点,否则造成泄漏。⚠️ 无限循环陷阱:遍历逻辑错误(如错误终止条件)易导致死循环。随机访问效率与单链表相同,均为 $O(n)$,不适合频繁随机查询的场景。\par
\chapter{实战应用案例:约瑟夫问题求解}
约瑟夫问题描述 $N$ 人围圈报数,每数到第 $K$ 人淘汰,求最后幸存者。循环链表提供优雅解法:\par
\begin{lstlisting}[language=c]
Node* josephus(int n, int k) {
    if (n < 1 || k < 1) return NULL;  // 边界检查

    // 构建循环链表:创建 n 个节点并成环
    Node* head = create_node(1);      // 头节点,数据为 1
    Node* prev = head;                // 前驱指针
    for (int i = 2; i <= n; i++) {
        prev->next = create_node(i);  // 添加新节点
        prev = prev->next;            // 更新前驱
    }
    prev->next = head;                // 尾节点指向头节点,闭环

    // 淘汰逻辑
    Node* current = head;
    while (current->next != current) {  // 终止条件:只剩一个节点
        // 移动 k-1 步(跳过 k-1 个节点)
        for (int i = 1; i < k-1; i++) {
            current = current->next;
        }
        // 删除第 k 个节点
        Node* temp = current->next;    // 临时保存待删除节点
        current->next = temp->next;    // 跳过待删除节点
        free(temp);                   // 释放内存
        current = current->next;       // 从下一节点继续
    }
    return current;  // 返回幸存者节点
}
\end{lstlisting}
代码解读:首先生成包含 $n$ 个节点的循环链表。淘汰阶段,每次移动 $k-1$ 步后删除第 $k$ 个节点。循环终止时仅剩一个节点,即幸存者。时间复杂度为 $O(n \times k)$,空间复杂度 $O(n)$。\par
\chapter{进阶讨论}
双向循环链表扩展了单向版本,每个节点包含 \texttt{prev} 和 \texttt{next} 指针。插入操作需同时维护双向闭环:\par
\begin{lstlisting}[language=c]
new_node->next = current->next;
new_node->prev = current;
current->next->prev = new_node;
current->next = new_node;
\end{lstlisting}
⚠️ 读者可尝试实现双向循环链表的删除操作,注意 \texttt{prev} 指针的更新。与数组实现的循环队列相比,循环链表在动态扩容上占优,但随机访问性能较差(数组为 $O(1)$,链表为 $O(n)$)。Linux 内核的 \texttt{list.h} 源码展示了工业级应用:通过宏定义实现高效通用的循环链表,支持进程调度和内存管理。\par
循环链表的适用场景可由决策树描述:若需高效头尾操作或连续遍历(如轮询系统),优先选择循环链表;若需随机访问,则考虑数组结构。关键学习收获是闭环思维在数据结构设计中的力量——通过消除边界,提升操作效率。延伸学习建议包括跳表(优化查询效率)和循环双端队列(结合队列与链表优势)。掌握这些概念,可深化对环形数据流处理的理解。\par
