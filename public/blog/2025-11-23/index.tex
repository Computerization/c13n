\title{着色器:仅用 x 和 y 坐标绘制高保真图形}
\author{黄京}
\date{Nov 23, 2025}
\maketitle
\chapter{探索片段着色器的核心思想,用数学公式代替像素画笔}
在计算机图形学中,着色器编程代表了一种革命性的思维方式:它允许开发者通过简洁的数学表达式来定义视觉元素,而非依赖预制的纹理或逐像素绘制。本文将引导您深入理解如何利用片段着色器,仅凭归一化的 x 和 y 坐标,创造出从基础几何到复杂特效的高保真图形。读完本文,您将掌握着色器编程的核心概念,并能亲手编写代码生成各类动态图形。\par
想象一个场景:复杂的渐变、分形图案或动态光晕效果,这些视觉元素并非来自加载的图片,而是由一个简单的数学函数根据每个像素的 x 和 y 坐标实时计算得出。这引出了一个根本性问题:我们习惯于在画布上绘制形状或加载图像,但如果能通过定义而非绘制来创造图形,会带来怎样的可能性?本文旨在解答这一问题,聚焦于片段着色器的应用,帮助您从传统图形处理转向数学驱动的视觉创作。通过本文的学习,您将能够理解着色器编程的思维模式,并独立实现各种图形效果。\par
\chapter{基础准备:我们的画布与坐标系}
在着色器编程中,片段着色器扮演着核心角色。与顶点着色器处理几何顶点不同,片段着色器负责为每个屏幕像素调用一次,决定其最终颜色。这使其成为图形定义的理想工具。为了通用性,我们引入归一化坐标的概念。通过将像素坐标 \texttt{fragCoord.xy} 除以画布尺寸 \texttt{iResolution.xy},我们得到归一化坐标 \texttt{uv},其范围在 [0, 1] 之间。这确保了代码在不同分辨率下的适应性。\par
例如,以下代码演示了如何计算归一化坐标:\par
\begin{lstlisting}[language=glsl]
vec2 uv = fragCoord.xy / iResolution.xy;
\end{lstlisting}
在这段代码中,\texttt{fragCoord.xy} 代表当前像素的坐标,而 \texttt{iResolution.xy} 是画布的宽度和高度。除法操作将坐标映射到 [0, 1] 区间,从而创建一个与分辨率无关的画布。为进一步优化,我们可以将坐标中心调整到画布中点,并修正宽高比以防止图形拉伸。进阶代码可能如下:\par
\begin{lstlisting}[language=glsl]
uv -= 0.5;
uv.x *= iResolution.x / iResolution.y;
\end{lstlisting}
这里,\texttt{uv -= 0.5} 将坐标原点移至画布中心,范围变为 [-0.5, 0.5]。接着,通过乘以宽高比,我们确保画布在 x 和 y 方向上比例一致,避免变形。这些步骤为后续图形定义奠定了坚实基础。\par
\chapter{核心武器:符号距离函数}
符号距离函数是着色器图形定义的核心工具。它本质上是一个数学函数 \texttt{f(point)},返回给定点到目标图形最近边缘的有符号距离。距离的正负值具有明确含义:正值表示点在形状外部,零值表示点在边界上,而负值表示点在形状内部。这类似于一个形状的引力场,负值区域定义了形状本身。\par
从符号距离函数到可视图形,我们需要借助步进函数。例如,\texttt{step(edge, x)} 函数在 x 小于 edge 时返回 0,否则返回 1,而 \texttt{smoothstep(edge0, edge1, x)} 提供平滑过渡,有助于抗锯齿。以绘制圆形为例,其符号距离函数公式为 $ \text{length}(uv) - r $,其中 $ r $ 是半径。在代码中,我们可以这样实现:\par
\begin{lstlisting}[language=glsl]
float d = length(uv) - radius;
float circle = step(d, 0.0);
\end{lstlisting}
在这段代码中,\texttt{length(uv)} 计算点到原点的距离,减去半径后得到有符号距离 \texttt{d}。\texttt{step(d, 0.0)} 将负值(内部)映射为 1(显示),正值(外部)映射为 0(隐藏),从而生成一个硬边缘圆形。若使用 \texttt{smoothstep},则可以创建软边缘效果:\par
\begin{lstlisting}[language=glsl]
float circle = smoothstep(0.01, 0.0, d);
\end{lstlisting}
这里,\texttt{smoothstep} 在距离接近零时平滑插值,减少锯齿现象。通过直接输出 \texttt{d} 作为灰度,我们还可以可视化距离场,直观理解形状的分布。\par
常见图形的符号距离函数构成了我们的工具箱。例如,圆形的公式为 $ \text{length}(p) - r $,矩形可使用 \texttt{box} 函数,直线可通过 \texttt{sdSegment} 实现。这些函数允许我们快速构建基础几何,而无需依赖外部资源。\par
\chapter{融合与组合:图形布尔运算}
符号距离函数的强大之处在于其支持布尔运算,从而组合复杂形状。通过数学操作,我们可以实现并集、交集和差集。并集使用 \texttt{min(d1, d2)} 函数,取两个距离场中的较小值,融合形状;交集使用 \texttt{max(d1, d2)},仅保留两个形状重叠部分;差集则通过 \texttt{max(d1, -d2)} 从第一个形状中减去第二个。\par
以绘制吃豆人图形为例,我们可以分步实现。首先,定义一个圆形的符号距离函数:\par
\begin{lstlisting}[language=glsl]
float circle_d = length(uv) - radius;
\end{lstlisting}
接着,定义一个矩形作为嘴巴部分:\par
\begin{lstlisting}[language=glsl]
float mouth_d = sdBox(uv, mouthSize);
\end{lstlisting}
其中 \texttt{sdBox} 是矩形的距离函数。最后,应用差集运算:\par
\begin{lstlisting}[language=glsl]
float pacman_d = max(circle_d, -mouth_d);
float final_shape = step(pacman_d, 0.0);
\end{lstlisting}
在这段代码中,\texttt{max(circle\_{}d, -mouth\_{}d)} 实现了从圆形中减去矩形的效果,因为 \texttt{-mouth\_{}d} 将矩形内部变为负值,外部变为正值,结合 \texttt{max} 操作后,仅当点在圆形内且不在矩形内时结果为负。\texttt{step} 函数将其转换为可视图形。这种方法展示了如何通过简单数学组合出复杂设计。\par
\chapter{超越几何:着色与质感}
一旦定义了几何形状,我们可以通过着色添加颜色和质感。动态颜色可以通过将坐标与颜色通道挂钩实现。例如,使用 \texttt{uv.x} 和 \texttt{uv.y} 驱动 RGB 值,创建线性渐变:\par
\begin{lstlisting}[language=glsl]
vec3 color = vec3(uv.x, uv.y, 0.5);
\end{lstlisting}
这段代码将 x 坐标映射为红色通道,y 坐标映射为绿色通道,生成一个从左上到右下的渐变。若结合三角函数,如 \texttt{sin} 和 \texttt{cos},可以创建条纹或波状图案:\par
\begin{lstlisting}[language=glsl]
float pattern = sin(uv.x * 10.0) * cos(uv.y * 10.0);
vec3 color = vec3(pattern);
\end{lstlisting}
这里,\texttt{sin} 和 \texttt{cos} 函数生成周期性变化,输出波状纹理。\par
为模拟光照效果,我们引入法线向量和漫反射模型。符号距离函数的梯度近似为法线,可通过 \texttt{fwidth} 函数计算:\par
\begin{lstlisting}[language=glsl]
vec3 normal = normalize(vec3(dfdx(d), dfdy(d), 1.0));
\end{lstlisting}
在这段代码中,\texttt{dfdx(d)} 和 \texttt{dfdy(d)} 计算距离场在 x 和 y 方向的偏导数,构成法线向量的 x 和 y 分量,z 分量设为 1.0 以标准化。接着,定义光源方向 \texttt{lightDir},并应用漫反射模型:\par
\begin{lstlisting}[language=glsl]
float diff = max(dot(normal, lightDir), 0.0);
vec3 color = vec3(diff);
\end{lstlisting}
\texttt{dot(normal, lightDir)} 计算法线与光源的点积,\texttt{max} 确保非负,生成亮度变化。这使得一个平面圆形呈现出立体球体效果。进一步结合镜面反射,可以创建金属质感:\par
\begin{lstlisting}[language=glsl]
float spec = pow(max(dot(reflectDir, viewDir), 0.0), 10.0);
vec3 final_color = diff + spec;
\end{lstlisting}
这里,\texttt{reflectDir} 是反射方向,\texttt{viewDir} 是视角方向,\texttt{pow} 函数增强高光强度。通过这些步骤,寥寥几行代码便能实现逼真的材质效果。\par
\chapter{进阶魔法:引入噪声与时间}
为增加图形的复杂性和动态性,我们可以引入噪声函数和时间变量。噪声函数,如 Simplex 或 Perlin 噪声,通过伪随机扰动打破规则性。例如,用噪声扰动圆形边界创建云朵效果:\par
\begin{lstlisting}[language=glsl]
float noise = snoise(uv * 10.0);
float cloud_d = length(uv) - radius + noise * 0.1;
float cloud = smoothstep(0.0, 0.01, cloud_d);
\end{lstlisting}
在这段代码中,\texttt{snoise} 是噪声函数,输出值用于调整距离场,生成不规则形状。类似地,噪声可用于模拟大理石或木材纹理,通过调制颜色或法线实现。\par
时间变量 \texttt{iTime} 允许图形动态变化。例如,创建一个脉冲发光的球体:\par
\begin{lstlisting}[language=glsl]
float pulse = sin(iTime) * 0.1;
float d = length(uv) - (radius + pulse);
float circle = smoothstep(0.0, 0.01, d);
\end{lstlisting}
这里,\texttt{sin(iTime)} 生成周期性变化,叠加到半径上,使球体大小随时间脉冲。旋转效果可通过修改坐标实现:\par
\begin{lstlisting}[language=glsl]
float angle = iTime;
vec2 rotated_uv = vec2(uv.x * cos(angle) - uv.y * sin(angle), uv.x * sin(angle) + uv.y * cos(angle));
\end{lstlisting}
这段代码应用旋转矩阵到坐标 \texttt{uv},生成动态旋转图形。这些技术将静态图形转化为生动动画,扩展了创作可能性。\par
回顾本文,归一化坐标 \texttt{uv} 作为万能画笔,符号距离函数作为图形定义语言,结合数学运算和 GLSL 内置函数,构成了着色器编程的核心框架。这种方法优势显著:图形基于数学定义,支持无限分辨率缩放;计算高度并行,契合 GPU 架构;动态和参数化调整简便,激发创意表达。\par
我们鼓励读者实践这些概念,从修改参数开始,逐步尝试组合不同符号距离函数,最终创造独特图形。资源如 Shadertoy 平台提供丰富示例,可供学习参考。在着色器的世界里,您不再是画家,而是世界的定义者——想象力是唯一的边界。通过持续探索,您将解锁更多视觉魔法,从简单公式中孕育出无限复杂。\par
