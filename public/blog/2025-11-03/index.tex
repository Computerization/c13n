\title{基数排序(Radix Sort)算法}
\author{杨其臻}
\date{Nov 03, 2025}
\maketitle
在计算机科学中,排序算法是基础且关键的主题。我们熟知的基于比较的排序算法,如快速排序和归并排序,其时间复杂度下界为 ( O(n \textbackslash{}log n) ),这意味着它们无法在比较模型下突破这一限制。然而,是否存在一种算法能够超越这一界限,在某些场景下实现更高效的排序呢?答案是肯定的——基数排序作为一种非比较型整数排序算法,能够在处理特定数据时达到 ( O(k \textbackslash{}times n) ) 的时间复杂度,其中 k 表示数字的最大位数。本文旨在深入解析基数排序的原理、实现方式及其性能,帮助读者从理论到实践全面掌握这一算法。\par
\chapter{理解基数排序的核心思想}
基数排序的核心在于“基数”这一概念。基数指的是进制的基数,例如十进制数的基数为 10,二进制数的基数为 2。算法通过按位排序来实现整体有序,排序顺序可以从最低有效位到最高有效位(LSD 方式),或反之(MSD 方式)。这种方法的巧妙之处在于,它避免了直接比较元素大小,而是通过分配和收集操作来逐步排序。\par
为了直观理解基数排序,让我们以一个简单的十进制数字数组为例,如 \texttt{[170, 45, 75, 90, 802, 24, 2, 66]}。排序过程从最低位(个位)开始,将数字分配到对应的桶中(0 到 9),然后按桶顺序收集回数组。接着处理十位数,重复分配和收集操作,最后处理百位数。每一轮排序都必须保持稳定性,即相等元素的相对顺序在排序后不变。稳定性是基数排序正确性的关键,因为如果某一轮排序不稳定,之前位数的排序信息可能会丢失,导致最终结果错误。\par
\chapter{基数排序的两种实现方式}
基数排序主要有两种实现方式:最低位优先(LSD)和最高位优先(MSD)。LSD 方式从数字的最低位开始排序,逐步向最高位推进。这种方式直观且易于实现,因为每一轮排序都基于前一轮的结果,通过稳定排序确保高位权重更大时能够覆盖低位的顺序。LSD 的正确性依赖于稳定排序的累积效应,最终实现整体有序。\par
相比之下,MSD 方式从最高位开始排序,并递归地对每个桶中的数字处理次高位。MSD 更类似于分治策略,可能在早期就将数据分割成小块,从而减少后续处理次数。然而,MSD 的实现较为复杂,需要处理递归和边界情况,通常更适用于字符串字典序排序等场景。尽管 LSD 和 MSD 都基于位的分配和收集,但它们在处理顺序和效率上各有侧重,读者可以根据具体需求选择适合的方式。\par
\chapter{手把手实现 LSD 基数排序}
LSD 基数排序的实现可以分为几个清晰步骤。首先,需要找到数组中的最大数字,以确定排序的轮数(即最大位数)。其次,初始化十个桶,对应数字 0 到 9。然后,从最低位开始,遍历每一位进行分配和收集操作:分配阶段将每个元素根据当前位数字放入对应桶中,收集阶段按桶顺序将元素取回数组。重复这一过程,直到处理完最高位。\par
以下是一个 Python 实现示例,我们将逐步解读代码关键部分。\par
\begin{lstlisting}[language=python]
def radix_sort(arr):
    # 步骤 1:找到最大值,确定最大位数
    max_num = max(arr)
    exp = 1 # 起始位:个位
    
    while max_num // exp > 0:
        # 步骤 2:初始化 10 个桶
        buckets = [[] for _ in range(10)]
        
        # 步骤 3a:分配
        for num in arr:
            digit = (num // exp) % 10
            buckets[digit].append(num)
        
        # 步骤 3b:收集
        arr_index = 0
        for bucket in buckets:
            for num in bucket:
                arr[arr_index] = num
                arr_index += 1
        
        # 移动到下一位
        exp *= 10

# 测试代码
if __name__ == "__main__":
    data = [170, 45, 75, 90, 802, 24, 2, 66]
    radix_sort(data)
    print("排序后的数组 :", data)
\end{lstlisting}
在这段代码中,首先通过 \texttt{max(arr)} 获取数组最大值,从而确定需要处理的位数。变量 \texttt{exp} 初始化为 1,表示从个位开始。循环条件 \texttt{max\_{}num // exp > 0} 确保在所有位数处理完毕前持续迭代。在每一轮中,初始化十个空桶用于存储数字。分配阶段使用 \texttt{(num // exp) \%{} 10} 计算当前位数字,这通过整数除法和取模操作实现,例如当 \texttt{exp} 为 1 时,\texttt{(num // 1) \%{} 10} 得到个位数;当 \texttt{exp} 为 10 时,得到十位数。数字被添加到对应桶中,这里使用列表作为桶,天然保证了稳定性,因为元素按添加顺序存储。收集阶段遍历所有桶,按顺序将元素放回原数组,索引 \texttt{arr\_{}index} 用于跟踪数组位置。最后,\texttt{exp} 乘以 10 移动到下一位。测试部分演示了算法对示例数组的排序效果,输出应为有序数组。\par
\chapter{深入分析与探讨}
基数排序的时间复杂度为 ( O(k \textbackslash{}times n) ),其中 k 是最大数字的位数,n 是数组长度。每一轮分配和收集操作各需 ( O(n) ) 时间,总共进行 k 轮。与基于比较的排序算法如快速排序的 ( O(n \textbackslash{}log n) ) 相比,当 k 小于 ( \textbackslash{}log n ) 时,基数排序可能更高效;但如果数字范围极大(k 很大),效率可能下降。空间复杂度为 ( O(n + r) ),其中 r 是基数(这里为 10),因为需要额外空间存储桶和元素。\par
基数排序的优点包括线性时间复杂度和稳定性,适用于固定位数的整数或字符串排序。然而,它并非原地排序,需要额外内存,且仅适用于可分割为位的数据类型。优化方面,可以选择不同基数(如 256 进制)以利用位运算加速,或使用链表优化桶操作减少数据搬移。\par
基数排序以其非比较的独特思想,在排序算法家族中占据重要地位。通过按位分配和收集,它实现了高效排序,特别适合处理手机号、身份证号等固定位数数据。理解并实现基数排序,不仅能扩展算法知识,还能在特定场景下提升应用性能。读者可以尝试用其他语言实现,或探索 MSD 版本以加深理解。\par
\chapter{互动与思考题}
动手实现基数排序是巩固知识的好方法,读者可以尝试用 Java 或 C++ 重写代码,或实现 MSD 版本。思考题包括:如何修改算法以处理负数?对于长度不一的字符串排序,该如何调整?为什么在现实应用中快速排序更常见?这些问题有助于进一步探索算法的边界和优化方向。\par
