\title{基数排序(Radix Sort)算法}
\author{马浩琨}
\date{Oct 06, 2025}
\maketitle
假设我们面对一个简单的问题:如何对数组 \texttt{[170, 45, 75, 90, 802, 24, 2, 66]} 进行排序?许多读者可能会立刻联想到快速排序、归并排序等经典的比较排序算法。这些算法通过直接比较元素的大小来决定它们的顺序,但它们在理论上的时间复杂度下界是 $O(n\log{n})$,这意味着在最坏情况下,排序 $n$ 个元素至少需要与 $n\log{n}$ 成正比的时间。然而,是否存在一种方法能够突破这个下界呢?答案是肯定的,基数排序就是一种非比较排序算法,它不依赖于元素之间的直接比较,而是通过逐位处理来实现排序。在特定条件下,基数排序可以达到线性时间复杂度 $O(n\cdot{k})$,其中 $k$ 是数字的最大位数。本文的目标是深入解析基数排序的核心思想,详细说明其工作流程,并提供可实现的代码示例,同时分析其优缺点和适用场景,帮助读者全面掌握这一算法。\par
\chapter{基数排序的核心思想解析}
基数排序的核心思想在于逐位排序。它将待排序元素视为由多个“位”组成的序列,例如整数可以分解为个位、十位、百位等。算法从最低有效位开始,依次对每一位进行排序,直到最高有效位。这一过程的关键在于每次排序必须是稳定的,即如果两个元素在某一位上相等,排序后它们的相对顺序保持不变。稳定性是基数排序能够正确工作的基石,因为它确保了高位排序的成果不会被低位的排序破坏。例如,假设我们先按十位排序数组 \texttt{[21, 11, 22, 12]},得到 \texttt{[11, 21, 12, 22]},其中 11 在 21 之前,12 在 22 之前。如果后续按个位排序时不稳定,11 和 21 的相对顺序可能被打乱,导致最终结果错误。稳定性保证了在逐位排序过程中,先前排序的顺序得以保留。\par
一个形象的比喻是整理扑克牌。想象你有多张扑克牌,需要先按花色排序,再按点数排序。首先,你将所有牌按花色分成四堆,然后在不打乱每堆内部顺序的前提下,再按点数排序。这个“保持原有顺序”的过程正是稳定性的体现。基数排序也类似,它是一种多关键字排序方法,通过逐位处理来构建最终的有序序列。\par
\chapter{算法流程详解:以 LSD 为例}
基数排序通常有两种实现方式:最低有效位优先和最高有效位优先。这里我们以最低有效位优先为例进行详细说明。假设我们使用示例数组 \texttt{[170, 45, 75, 90, 802, 24, 2, 66]}。首先,我们需要找到数组中的最大值,以确定排序的轮数。最大值是 802,它有 3 位数字,因此我们需要进行 3 轮排序,分别对应个位、十位和百位。\par
第一轮排序针对个位。我们创建 10 个桶,对应数字 0 到 9。将每个数字按其个位数放入对应的桶中,例如 170 的个位是 0,放入 0 号桶;45 的个位是 5,放入 5 号桶。完成分配后,我们按顺序从桶 0 到桶 9 收集数字,形成新的数组。此时,数组按个位有序,但整体可能仍未排序。\par
第二轮排序针对十位。我们对上一轮得到的新数组,根据十位数进行分配。需要注意的是,对于位数不足的数字,如 2,其十位视为 0。在分配过程中,稳定性至关重要。例如,在个位排序后的数组中,170 和 90 的十位都是 7,稳定性确保了 170 依然在 90 之前。分配完成后,再次按顺序收集数字。\par
第三轮排序针对百位。同样地,我们根据百位数进行分配和收集。经过这三轮排序,数组最终完全有序。整个过程通过逐位处理,利用稳定性保证了排序的正确性,而无需直接比较元素大小。\par
\chapter{代码实现:以 Python 为例}
下面我们提供一个基数排序的 Python 实现代码,并详细解读每一步。该代码针对非负整数设计,后续我们会讨论如何处理负数。\par
\begin{lstlisting}[language=python]
def radix_sort(arr):
    if len(arr) < 2:
        return arr

    max_val = max(arr)
    exp = 1

    while max_val // exp > 0:
        buckets = [[] for _ in range(10)]

        for num in arr:
            digit = (num // exp) % 10
            buckets[digit].append(num)

        arr_index = 0
        for bucket in buckets:
            for num in bucket:
                arr[arr_index] = num
                arr_index += 1

        exp *= 10

    return arr
\end{lstlisting}
首先,函数检查数组长度是否小于 2,如果是,则直接返回,因为单个元素或空数组已经有序。接下来,找到数组中的最大值 \texttt{max\_{}val},以确定排序的轮数。变量 \texttt{exp} 初始化为 1,代表当前处理的位数(从个位开始)。\par
在循环中,只要 \texttt{max\_{}val // exp} 大于 0,就继续排序。每一轮循环中,我们初始化 10 个空桶,对应数字 0 到 9。然后遍历数组,对每个数字计算当前位的值,使用表达式 \texttt{(num // exp) \%{} 10}。例如,当 \texttt{exp} 为 1 时,这计算个位;当 \texttt{exp} 为 10 时,计算十位。数字被放入对应的桶中。\par
收集过程按桶的顺序(0 到 9)进行,将每个桶中的数字依次放回原数组。这一步保证了排序的稳定性,因为桶内元素的顺序保持不变。最后,\texttt{exp} 乘以 10,移动到下一位,循环继续直到处理完所有位。\par
对于负数的处理,我们可以将数组分成负数和非负数两部分。对负数部分取绝对值,进行基数排序后反转顺序;对非负数部分直接排序;最后合并两部分。这扩展了算法的适用性,但需要额外注意边界情况。\par
\chapter{算法分析}
基数排序的时间复杂度为 $O(k\cdot n)$,其中 $k$ 是最大数字的位数,$n$ 是数组长度。算法需要进行 $k$ 轮排序,每轮包括分配和收集两个步骤,每个步骤的时间复杂度为 $O(n)$,因此总时间为 $O(k \cdot n)$。当 $k$ 远小于 $n$ 时,基数排序的性能优于比较排序的 $O(n\log{n})$ 下界。\par
空间复杂度为 $O(n + r)$,其中 $r$ 是基数的大小(这里 $r = 10$)。算法需要额外的空间来存储 $r$ 个桶和桶中的 $n$ 个元素。基数排序是稳定的排序算法,这在多关键字排序中尤为重要。\par
\chapter{优缺点与应用场景}
基数排序的主要优点在于其速度,尤其在数据范围不大且数据量巨大时,它可以实现线性时间复杂度。此外,它的稳定性使其适用于需要保持相对顺序的场景。然而,基数排序也有明显的缺点:它需要额外的内存空间,且对数据格式有严格要求,通常只适用于整数或可以分解为“位”的结构。如果数据范围很大($k$ 很大),效率会显著下降。\par
在实际应用中,基数排序常用于对电话号码、身份证号等固定位数的数字进行排序。它还在后缀数组构造和计算机图形学中的某些算法中发挥作用。选择基数排序时,需权衡其线性时间优势与空间开销及数据限制。\par
\chapter{进阶与变种}
除了最低有效位优先的实现,基数排序还有最高有效位优先的变种。最高有效位优先从最高位开始排序,通常需要递归处理,可能在中间就完成排序,但实现更复杂且不稳定。另一种变种是改变基数,例如使用 2 的幂次(如 256 进制)作为基数,这可以通过位运算快速获取“位”,但会增加桶的数量,影响空间效率。这些进阶内容为算法优化提供了更多可能性。\par
基数排序通过逐位排序和稳定性的结合,实现了在特定条件下的线性时间复杂度。它的核心思想简单而强大,但适用场景有限。读者在理解本文内容后,可以尝试亲手实现代码,并扩展处理负数的情况,以加深对非比较排序的理解。基数排序虽然不是万能算法,但在合适的数据集上,它无疑是一把高效的排序利器。\par
