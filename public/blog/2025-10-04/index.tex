\title{基本的双向文本(BiDirectional Text)处理机制}
\author{叶家炜}
\date{Oct 04, 2025}
\maketitle
\chapter{从乱码到清晰:揭秘阿拉伯语与希伯来语文本的渲染逻辑}
想象一下,你在一个简单的文本编辑器中输入字符串 \texttt{"Hello - 123 - عالم"},期望看到清晰的显示,但结果却可能是一片混乱:阿拉伯语部分顺序错乱,数字和标点符号位置不当。这种问题并非偶然,而是源于不同语言书写方向的冲突。英语和中文等语言遵循从左向右的书写顺序,而阿拉伯语和希伯来语等语言则采用从右向左的顺序。当这些方向混合时,如果没有适当的处理机制,渲染就会失败。这正是双向文本处理所要解决的核心问题。本文将带你深入探讨双向文本的复杂性,并亲手实现一个简化版的 Unicode 双向算法,让你从理论到实践全面掌握这一关键技术。\par
\section{背景知识}
双向文本指的是包含从左向右和从右向左混合书写方向的文本内容。例如,LTR 语言如英语、中文和俄语,其字符顺序从左侧开始向右延伸;而 RTL 语言如阿拉伯语和希伯来语,则从右侧开始向左书写。如果不加处理,简单地将这些文本拼接在一起,会导致显示顺序与逻辑顺序脱节,从而产生乱码。逻辑顺序指的是文本在内存中的存储序列,而显示顺序则是最终在屏幕上呈现的视觉序列。Unicode 标准通过其双向算法规范 UAX \#{}9 来解决这一问题,它为文本渲染引擎提供了一套规则,确保混合方向文本的正确显示。\par
\section{核心原理}
Unicode 双向算法是处理双向文本的核心,它分为三个阶段:分解段落、解析与重置、以及重新排序与镜像。首先,在阶段一,算法将输入文本按段落分隔符如换行符拆分成独立段落,并确定每个段落的基础方向。基础方向可以通过启发式规则基于首强字符推断,或由外部协议如 HTML 的 \texttt{dir} 属性指定。强字符包括 L、R 和 AL 等类别,它们具有明确的方向性;弱字符如数字和标点符号方向性较弱;中性字符如空格和分隔符则依赖上下文确定方向。\par
阶段二是算法的核心,涉及解析字符方向并重置层级。层级是一个整数,0 表示 LTR 基础方向,1 表示 RTL,以此类推。算法使用栈结构处理显式嵌入字符如 RLE 和 LRE,以及重写字符如 LRO 和 RLO,这些字符可以临时改变文本方向。例如,RLE 字符会推入一个 RTL 嵌入级别到栈中,直到遇到 PDF 字符才弹出。接下来,算法解析中性字符,根据规则 N1 和 N2 确定其方向:N1 规则要求中性字符继承前一个强字符的方向,如果不存在,则继承段落基础方向;N2 规则处理数字周围的中性字符,确保它们与数字方向一致。举例来说,一个句号 \texttt{.} 在英文和阿拉伯文混合文本中,会根据相邻强字符决定其显示位置。最后,规则 L1 处理数字,确保在 RTL 段落中数字仍保持 LTR 内部顺序,避免顺序混乱。\par
阶段三负责重新排序和字符镜像。根据计算出的层级,算法使用反转层级方法:偶数层级从左向右显示,奇数层级从右向左显示。同时,字符镜像会将对称字符如括号 \texttt{()} 在 RTL 上下文中替换为镜像形式 \texttt{)(},以保持视觉一致性。这三个阶段共同确保文本从逻辑顺序正确映射到显示顺序。\par
\section{动手实践}
现在,我们来动手实现一个简化的双向文本处理算法。这个版本专注于纯文本处理,忽略显式嵌入字符和数字形状处理,输入为一个字符串和段落基础方向,输出重新排序后的字符数组。我们选择 Python 作为实现语言,因其简洁性和丰富的 Unicode 支持。首先,我们需要获取字符的双向类别,可以使用 Python 的 \texttt{unicodedata} 库来查询 \texttt{bidi\_{}class} 属性。\par
在步骤一中,我们设置环境并定义字符方向性查找函数。以下代码初始化一个字典,映射常见字符到其双向类别,但为了简化,我们硬编码一些示例字符的类别。例如,英文字母归类为 L,阿拉伯字母归类为 AL,数字为 EN,标点为 ON。\par
\begin{lstlisting}[language=python]
import unicodedata

def get_bidi_class(char):
    return unicodedata.bidi_class(char)
\end{lstlisting}
这段代码使用 \texttt{unicodedata.bidi\_{}class} 函数获取任意 Unicode 字符的双向类别。例如,字符 'A' 的类别是 'L',表示从左向右;字符 'ع' 的类别是 'AL',表示阿拉伯字母从右向左。通过这个函数,我们可以为每个字符分配初始方向属性,为后续解析奠定基础。\par
步骤二实现方向解析,遍历字符串并为每个字符分配方向,同时处理中性字符。我们根据规则 N1 和 N2 调整中性字符的方向:向前和向后查找最近的强字符,如果找到,则继承其方向;否则使用段落基础方向。\par
\begin{lstlisting}[language=python]
def resolve_neutral_chars(text, base_dir):
    chars = list(text)
    bidi_classes = [get_bidi_class(c) for c in chars]
    for i, char_class in enumerate(bidi_classes):
        if char_class == 'ON':  # 中性字符
            left_strong = None
            right_strong = None
            # 向左查找强字符
            for j in range(i-1, -1, -1):
                if bidi_classes[j] in ['L', 'R', 'AL']:
                    left_strong = bidi_classes[j]
                    break
            # 向右查找强字符
            for j in range(i+1, len(bidi_classes)):
                if bidi_classes[j] in ['L', 'R', 'AL']:
                    right_strong = bidi_classes[j]
                    break
            # 应用规则 N1
            if left_strong and right_strong:
                if left_strong == right_strong:
                    bidi_classes[i] = left_strong
                else:
                    bidi_classes[i] = base_dir
            elif left_strong:
                bidi_classes[i] = left_strong
            elif right_strong:
                bidi_classes[i] = right_strong
            else:
                bidi_classes[i] = base_dir
    return bidi_classes
\end{lstlisting}
这段代码首先将输入字符串转换为字符列表,并获取每个字符的双向类别。然后,它遍历每个字符,如果字符是中性类别 ON,则向前和向后搜索最近的强字符 L、R 或 AL。根据规则 N1,如果左右强字符方向一致,则中性字符继承该方向;如果不一致或找不到强字符,则使用段落基础方向。这确保了中性字符如标点符号在混合文本中正确对齐。\par
步骤三实现重新排序,根据层级将文本分成多个运行,并对奇数层级运行进行反转。我们假设所有字符初始层级为 0 或 1,基于段落基础方向。\par
\begin{lstlisting}[language=python]
def reorder_bidi_text(text, base_dir):
    base_level = 0 if base_dir == 'L' else 1
    levels = [base_level] * len(text)
    # 简化处理:假设无显式嵌入,因此层级不变
    runs = []
    current_run = []
    current_level = base_level
    for i, char in enumerate(text):
        if levels[i] != current_level:
            if current_run:
                runs.append((current_level, ''.join(current_run)))
            current_run = [char]
            current_level = levels[i]
        else:
            current_run.append(char)
    if current_run:
        runs.append((current_level, ''.join(current_run)))
    # 反转奇数层级运行
    reordered_chars = []
    for level, run in runs:
        if level % 2 == 1:  # 奇数层级,从右向左
            reordered_chars.extend(list(reversed(run)))
        else:  # 偶数层级,从左向右
            reordered_chars.extend(list(run))
    return ''.join(reordered_chars)
\end{lstlisting}
这段代码首先初始化层级数组,假设所有字符具有相同的层级(基于段落基础方向)。然后,它将文本分成连续运行,每个运行包含相同层级的字符。对于奇数层级运行,代码使用 \texttt{reversed} 函数反转字符顺序,模拟从右向左显示;偶数层级运行保持原顺序。最后,将所有运行拼接成最终字符串。例如,输入字符串 \texttt{"car (CAR) car"} 在 RTL 基础方向下,输出应为 \texttt{"rac (RAC) rac"},因为括号内的部分在奇数层级被反转。\par
为了测试我们的实现,我们可以运行一个示例:给定输入 \texttt{"Hello - 123 - عالم"} 和基础方向 LTR,算法应正确解析中性字符并重新排序 RTL 部分。通过打印中间步骤如解析后的方向和最终输出,我们可以验证算法的正确性。\par
\section{现实世界的应用}
在实际应用中,双向文本处理广泛集成于现代技术中。例如,在 HTML 和 CSS 中,可以使用 \texttt{dir} 属性指定文本方向,配合 \texttt{unicode-bidi} 和 \texttt{direction} 属性实现复杂渲染。文本引擎如 HarfBuzz 和 Pango 实现了完整的 UAX \#{}9 算法,支持多语言文本布局。终端和编辑器如 VS Code 也内置了双向文本支持,确保代码和注释在混合语言环境下的可读性。需要注意的是,我们的简化实现忽略了显式嵌入字符和数字处理,完整版本涉及更多边界情况和优化,但这些库提供了可靠的生产级解决方案。\par
通过本文,我们深入探讨了双向文本的核心问题,从乱码现象出发,解析了 Unicode 双向算法的三个阶段,并亲手实现了一个简化版本。关键收获在于理解了逻辑顺序与显示顺序的映射关系,以及算法如何通过层级和方向解析确保文本正确渲染。尽管我们的实现侧重于基础功能,但它为进一步探索完整规范奠定了基础。鼓励读者阅读 UAX \#{}9 官方文档,或尝试集成成熟库如 HarfBuzz 到实际项目中,以解决更复杂的双向文本挑战。\par
\section{参考资料}
Unicode 官方标准 UAX \#{}9 提供了双向算法的完整规范,可在 Unicode 官网查阅。W3C 关于双向文本的指南提供了 Web 开发中的实践建议。HarfBuzz 和 Pango 项目是开源文本布局引擎,源码可用于深入学习。此外,Unicode 联盟提供了测试字符串,可用于验证算法实现。这些资源将帮助你扩展知识并应用于实际场景。\par
