\title{"深入理解基数排序"}
\author{"杨子凡"}
\date{"Jun 08, 2025"}
\maketitle
排序算法在计算机科学中占据基础地位,广泛应用于数据库查询优化、大数据处理系统等场景。基数排序作为一种非比较排序算法,具有独特的优势:它能够突破基于比较的排序算法(如快速排序或归并排序)的时间复杂度下界 O(n log n),在特定场景下实现线性时间复杂度。本文将从底层原理出发,结合 Python 代码实现,深入解析基数排序的工作机制、性能特征及优化策略,帮助读者透彻理解其适用边界与实现细节。\par
\chapter{基数排序基础概念}
基数排序的核心思想是“按位分组排序”,即通过多轮分配与收集操作,从最低位(LSD)或最高位(MSD)开始逐位处理元素。这本质上是桶排序的扩展,利用稳定排序的叠加效应实现全局有序。关键术语包括「数位」——指元素的每一位(如数字的个位、十位),「基数」——代表数位的取值范围(如十进制基数为 10),以及排序方向的选择:LSD(Least Significant Digit)从最低位开始排序,适用于大多数整数场景;MSD(Most Significant Digit)从最高位开始,行为类似字典树,常用于字符串排序。\par
\chapter{算法原理深度剖析}
LSD 基数排序的流程分为三步:首先计算最大数字的位数 k;然后从最低位到最高位遍历,每轮按当前位分配桶(桶数量等于基数),再按桶顺序收集元素;最终返回有序数组。稳定性在此至关重要——例如对数组 \texttt{[21, 15, 12]} 按十位排序后(桶分组为 \texttt{[12]} 和 \texttt{[21, 15]}),个位排序需保持 \texttt{21} 与 \texttt{15} 的相对顺序,否则结果错误。时间复杂度为 O(d \textbackslash{}times (n + b)),其中 d 是最大位数,n 是元素数量,b 是基数。当 d 较小且 n 较大时(如处理 10\^{}7 个 32 位整数),基数排序性能优于快排等算法,因为其避免了比较操作的 log n 因子。\par
\chapter{代码实现(Python 示例)}
\begin{lstlisting}[language=python]
def radix_sort(arr):
    # 计算最大数字的位数
    max_num = max(arr)
    exp = 1
    while max_num // exp > 0:
        counting_sort(arr, exp)
        exp *= 10

def counting_sort(arr, exp):
    n = len(arr)
    output = [0] * n
    count = [0] * 10  # 十进制基数
    
    # 统计当前位出现次数
    for i in range(n):
        index = arr[i] // exp % 10
        count[index] += 1
    
    # 计算累计位置
    for i in range(1, 10):
        count[i] += count[i-1]
    
    # 按当前位排序(逆序保证稳定性)
    i = n - 1
    while i >= 0:
        index = arr[i] // exp % 10
        output[count[index] - 1] = arr[i]
        count[index] -= 1
        i -= 1
    
    # 写回原数组
    for i in range(n):
        arr[i] = output[i]
\end{lstlisting}
这段代码实现 LSD 基数排序,核心是 \texttt{radix\_{}sort} 函数与子过程 \texttt{counting\_{}sort}。首先,\texttt{radix\_{}sort} 计算最大数字的位数(如 123 的位数为 3),通过变量 \texttt{exp}(初始为 1)控制当前处理的数位(个位、十位等)。\texttt{exp} 在每轮乘以 10,直到覆盖最高位。子过程 \texttt{counting\_{}sort} 是桶排序的优化版本:它统计当前位(由 \texttt{exp} 指定)的出现频率,计算累计位置以确定元素在输出数组中的索引。关键点在于逆序填充(从数组末尾开始处理),这保证了稳定性——当两个元素当前位相同时,原始顺序得以保留。例如,在十位排序轮中,\texttt{21} 和 \texttt{15} 同属桶 1,逆序处理确保 \texttt{21} 先于 \texttt{15} 被放置,维持相对位置。最后,排序结果写回原数组。\par
\chapter{关键问题与优化}
处理负数时有两种常见方案:一是分离正负数组,负数取绝对值排序后反转再合并;二是通过偏移量法将所有数加上最小值转为非负。对于字符串排序,可应用 LSD 方法从右到左按字符分桶(不足位补空字符),例如对 \texttt{["apple", "banana", "cherry"]} 排序时,首轮按末字符分桶。基数选择优化需权衡轮数与桶数:二进制基数(b=2)减少桶数但增加轮数(d 增大),十进制基数(b=10)则相反。理论最优点在 b ≈ n 时达到时间复杂度平衡,参考《算法导论》(CLRS)第 8 章分析。\par
\chapter{性能对比与局限}
基数排序在均匀分布的整数或定长字符串(如手机号、IP 地址)场景下表现卓越,尤其当数据量远大于数值范围时(例如 1e6 个 [0, 1000] 的整数)。然而,其空间复杂度 O(n + b) 带来显著桶开销,且不适合浮点数(需处理 IEEE 754 编码)或动态数据结构如链表。基准测试显示,在 1e7 个 32 位整数排序中,基数排序性能稳定,而快速排序在退化情况下(如已有序数组)效率骤降。\par
\chapter{应用案例}
在数据库索引优化中,基数排序支持多字段排序(如按年、月、日分层处理)。MapReduce 模型下可扩展为分布式版本,按数位分片并行处理。计算机图形学中的深度缓冲排序(Z-buffering)也依赖类似机制高效管理像素深度。\par
基数排序的本质是多轮稳定桶排序与数位分解思想的结合,但其“非比较排序”特性并非万能——需严格匹配数据特性(如元素可分割为离散位)。思考题:对数组 \texttt{[("Alice", 25), ("Bob", 20), ("Alice", 20)]} 按姓名→年龄排序时,可先按年龄(低位)稳定排序,再按姓名(高位)排序,利用稳定性保持同名元素的年龄顺序。\par
