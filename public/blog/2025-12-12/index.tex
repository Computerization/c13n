\title{Tokenization 在 NLP 中的工作原理}
\author{杨子凡}
\date{Dec 12, 2025}
\maketitle
想象一下,你输入一句简单的英文「Hello, world!」到 NLP 模型中,它如何理解这个句子?首先,这段文本会被拆分成一个个小单元,比如「Hello」、「,」、「world」、「!」这样的 token。这些 token 就像语言的积木块,是人类自然语言转化为机器可处理的数字序列的桥梁。Tokenization 作为 NLP 管道的第一步,直接决定了后续 embedding 和模型推理的质量。如果 tokenization 出问题,整个模型的性能都会受影响,比如罕见词汇无法正确拆分,导致准确率下降。\par
在 NLP 的核心流程中,文本首先经过 tokenization 转换为 token 序列,然后映射到词汇表 ID,再通过 embedding 层变成向量,最后输入 Transformer 等模型进行处理。这个过程看似简单,却至关重要:它影响计算效率,因为模型有最大序列长度限制;也影响准确率,因为好的 tokenization 能更好地捕捉语义,比如处理「人工智能」这样的中文词时,需要考虑无空格特性。本文将深入解释 tokenization 的原理、类型、算法、挑战及实际应用,面向初学者到中级开发者,提供 Python 代码示例,帮助你从零掌握这项基础技术。\par
通过阅读,你将理解为什么 BERT 和 GPT 使用不同的 tokenizer,以及如何在自己的项目中训练自定义 tokenizer。让我们从基础开始,一步步揭开这个 NLP 基石的神秘面纱。你准备好探索了吗?\par
\chapter{什么是 Tokenization?}
Tokenization 的本质是将原始文本拆分成更小的单元,这些单元称为 token,可以是完整的单词、子词片段,甚至单个字符。这个过程解决了 NLP 模型的一个根本问题:Transformer 等神经网络只能处理数字序列,而非人类语言的连续字符串。通过 tokenization,文本被转化为固定词汇表中的 ID 序列,便于后续向量化。\par
为什么需要 tokenization?因为直接用字符序列会让序列过长,计算开销巨大;用完整单词又会遇到 OOV(Out-of-Vocabulary)问题,即训练时未见过的词无法处理。Tokenization 巧妙平衡了这两者,提供了一个高效的桥梁。例如,在句子「Don't stop!」中,单词级 tokenization 可能输出「Don't」、「stop」、「!」,而子词级则进一步拆成「Don」、「'」、「t」、「stop」、「!」。\par
Token 的类型多样化,以适应不同场景。单词级按空格和标点拆分,简单直观,但对新词不友好;子词级如 BPE 将词拆成常见子单元,处理 OOV 更好;字符级则最细粒度,每个字符一个 token,灵活但序列长。在实际模型中,还有特殊 token 增强功能,比如 BERT 中的 [CLS] 用于分类任务的聚合表示,[SEP] 分隔句子,[PAD] 填充序列到固定长度,[UNK] 代表未知 token。这些特殊标记确保输入标准化,提高模型鲁棒性。\par
文本经过 tokenization 后,会映射到唯一 ID,比如词汇表大小为 30k 的模型中,「hello」可能对应 ID 101,然后生成 attention mask 区分真实 token 和填充部分。这个流程可视化为:原始文本 → token 列表 → ID 序列 → 模型输入。你知道自己的 NLP 项目中,tokenization 如何影响结果吗?\par
\chapter{Tokenization 的工作原理}
Tokenization 的核心流程分为几个步骤,首先是预处理,包括小写转换、标点规范化以及 Unicode 标准化,以减少噪声。然后是拆分,使用规则或统计模型将文本切分成 token。接着,词汇表映射将每个 token 转为唯一 ID,通常词汇表大小在 30k 到 100k 之间。最后是编码,生成 ID 序列和 attention mask;解码则是逆过程,从 ID 还原文本。\par
让我们通过 Python 示例直观理解,使用 Hugging Face 的 BertTokenizer:\par
\begin{lstlisting}[language=python]
from transformers import BertTokenizer

tokenizer = BertTokenizer.from_pretrained('bert-base-uncased')
tokens = tokenizer.tokenize("Hello, world!")
ids = tokenizer.convert_tokens_to_ids(tokens)
print(tokens)  # 输出 : ['hello', ',', 'world', '!']
print(ids)     # 输出 : [10176, 1010, 2088, 10008](实际 ID 依模型而定)
\end{lstlisting}
这段代码首先加载预训练的 BERT 分词器,\texttt{from\_{}pretrained} 从 Hugging Face Hub 下载 bert-base-uncased 模型的 tokenizer 配置,包括词汇表文件。\texttt{tokenizer.tokenize("Hello, world!")} 执行核心拆分:它先规范化文本(小写、去除多余空格),然后用 WordPiece 算法拆分成子词 token,输出 ['hello', ',', 'world', '!']。注意逗号独立成 token,这是为了捕捉标点语义。接着 \texttt{convert\_{}tokens\_{}to\_{}ids} 查询词汇表,将每个 token 映射到整数 ID,比如 'hello' 可能为 10176。这些 ID 是模型实际输入,加上特殊 token 如 [CLS] 和 [SEP] 后,形成完整序列。这个示例展示了从文本到数字的端到端转换,实际使用时还需调用 \texttt{tokenizer.encode\_{}plus} 添加 mask 和截断。\par
词汇表构建是从大规模语料库中训练而来:统计高频 token,设置频率阈值或采样低频部分,避免词汇表爆炸。训练过程迭代优化,确保常见词完整表示,罕见词拆分成子词。这个原理确保了 tokenization 的高效性和泛化能力。你试过调试自己的 tokenizer 输出吗?\par
\chapter{常见 Tokenization 方法与算法}
规则-based 方法是最基础的,比如单词级 tokenization 依赖空格和正则表达式拆分,NLTK 的 word\_{}tokenize 就是典型实现。它简单快速,适合英文,但 OOV 问题突出:新词如「COVID-19」只能用 [UNK] 表示。\par
统计-based 方法更强大,其中 Byte-Pair Encoding(BPE)被 GPT 系列广泛采用。BPE 算法从字符级开始,迭代合并语料中最频字符对构建子词词汇表。具体步骤:初始化每个字符为 token,统计相邻对频率,重复合并最高频对,直至达到词汇表大小。例如,从「low lower lowest」开始,第一轮可能合并「l o」成「lo」,逐步形成「low」、「lowest」等子词。这种贪婪合并高效处理 OOV,「un」+「known」可拆成已知子词。\par
WordPiece 是 BERT 的选择,类似 BPE 但在合并时选择最大 likelihood 提升的子词,提高 perplexity。Unigram Language Model 则相反,从大词汇表开始,概率采样删除低频 token 直到目标大小,SentencePiece 常用此法支持无空格语言如中文。\par
这些方法的对比鲜明:BPE 高效处理 OOV,需训练语料;WordPiece 优化语言建模但计算密集;SentencePiece 多语言友好但词汇表较大。下面是 BPE 简化实现演示:\par
\begin{lstlisting}[language=python]
def simple_bpe(texts, num_merges=10):
    words = [list(w) for w in texts.split()]  # 初始化字符列表
    merges = {}
    for i in range(num_merges):
        pairs = {}
        for word in words:
            for j in range(len(word)-1):
                pair = (word[j], word[j+1])
                pairs[pair] = pairs.get(pair, 0) + 1
        if not pairs:
            break
        best_pair = max(pairs, key=pairs.get)
        merges[best_pair] = i
        new_words = []
        for word in words:
            new_word = word
            while ''.join(new_word[:-1]) in merges:  # 贪婪合并
                new_word = new_word[:-2] + [''.join(new_word[-2:])]
            new_words.append(new_word)
        words = new_words
    return merges, words

merges, tokenized = simple_bpe("low lower lowest", 5)
print(merges)  # 输出类似 {(('l', 'o'), 0), (('lo', 'w'), 1)} 等合并对
print(tokenized)  # 输出子词序列
\end{lstlisting}
这段代码模拟 BPE 训练:\texttt{texts.split()} 分词成字符列表,循环统计相邻对频率,选最高频合并并记录在 \texttt{merges} 字典。合并后用贪婪方式应用到所有词,确保子词一致。这个简化版忽略了完整词汇表构建,但捕捉了核心迭代逻辑。在 Hugging Face 中,你可比较不同 tokenizer:\texttt{tokenizer("Hello")} 输出差异揭示算法特性,如 BPE 更倾向子词拆分。BPE 的合并过程就像逐步构建拼图,高效捕捉语言规律。你最喜欢哪种方法,为什么?\par
\chapter{Tokenization 在 NLP 模型中的作用}
在 Transformer 架构中,tokenization 位于输入 embedding 层之前,直接塑造向量表示的质量。没有它,模型无法处理变长序列。生成 token ID 后,加入位置编码(Positional Encoding),如 $\sin$ 和 $\cos$ 函数注入顺序信息:$PE(pos, 2i) = \sin(pos / 10000^{2i/d})$,确保模型感知位置。\par
实际影响显著:模型有序列长度限制,如 BERT 的 512 token,超长需截断或填充 [PAD],attention mask 屏蔽无效部分。多语言场景下,子词 tokenizer 处理中文「人工智能」为「人」、「工」、「智」、「能」,无空格依赖强。BERT 用 WordPiece 偏好完整英文词,GPT 的 BPE 更碎片化利于生成。\par
案例中,BERT tokenizer 保留标点独立,适合分类;GPT 优化连续生成。长文本如 Longformer 用滑动窗口 tokenization,动态调整 attention,减少 token 数。Transformer 输入管道从 tokenization 开始,串联 embedding 和模型,任何环节偏差都放大误差。思考你的模型输入如何优化?\par
\chapter{挑战与解决方案}
Tokenization 面临 OOV 问题,未见词用 [UNK] 表示,语义丢失严重,子词方法如 BPE 通过拆分解决,将「neuralink」拆成「neu」、「ralink」。长序列超过 max\_{}length 时,需智能截断保留关键部分,或用层次 tokenization 分块处理。\par
多语言尤其是低资源语言挑战大,英文依赖空格,中文无此特性,导致过拆;SentencePiece 直接处理原始文本,训练联合模型缓解。噪声文本如表情「😂」、URL「https://example.com」、拼写错误需自定义预处理规则,先规范化再 tokenization。\par
性能优化关键,TikToken 为 GPT 设计,用 Rust 实现超快编码,基准测试显示比 Python tokenizer 快 10 倍。这些解决方案让 tokenization 更鲁棒。你遇到过哪些 tokenization 坑?\par
\chapter{实际应用与工具推荐}
在情感分析中,tokenization 确保「I love AI!!」拆成捕捉强调的 token;在机器翻译,子词对齐源语和目标语;在聊天机器人,快速 tokenizer 支撑实时响应。流行库中,Hugging Face Tokenizers 最全面,支持 BPE 等训练自定义模型:\par
\begin{lstlisting}[language=python]
from tokenizers import Tokenizer, models, trainers, pre_tokenizers, decoders

tokenizer = Tokenizer(models.BPE())
tokenizer.pre_tokenizer = pre_tokenizers.Whitespace()
trainer = trainers.BpeTrainer(vocab_size=30000, special_tokens=["[UNK]", "[PAD]"])
files = ["corpus.txt"]  # 你的语料文件
tokenizer.train(files, trainer)
output = tokenizer.encode("人工智能很强大")
print(output.tokens)  # 输出子词如 ['人工', '智能', '很', '强大']
\end{lstlisting}
这段代码训练自定义 BPE:初始化 BPE 模型,用 Whitespace 预分词,Trainer 设置词汇表大小和特殊 token。\texttt{tokenizer.train} 从语料统计合并对,输出 tokenizer 对象。\texttt{encode} 处理新文本,展示子词拆分。这个教程让你 5 分钟上手自定义 tokenizer。NLTK/spaCy 适合规则入门,TikToken 专为 OpenAI 优化速度。基准显示 TikToken 每秒处理 100 万 token。动手试试吧!\par
\chapter{结论与展望}
Tokenization 从规则拆分演进到 BPE、WordPiece 等统计算法,成为 NLP 管道基石,桥接人类语言与数字模型,影响一切从 embedding 到推理的表现。\par
未来,动态 tokenization 按上下文自适应拆分,稀疏 tokenizer 压缩 token 数提升效率,多模态版本融合文本图像 token。行动起来:实验 Hugging Face 代码,分享你的 tokenizer 项目!\par
资源推荐:BPE 原论文「Neural Machine Translation of Rare Words with Subword Units」(arXiv:1508.07909),Hugging Face Tokenizers GitHub,tiktoken 在线 demo。你准备好构建下一个 NLP 项目了吗?\par
