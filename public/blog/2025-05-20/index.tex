\title{"从零实现一个简单的全文搜索引擎"}
\author{"杨子凡"}
\date{"May 20, 2025"}
\maketitle
在信息爆炸的互联网时代,全文搜索引擎已成为处理海量文本数据的核心工具。从 Google 的网页搜索到 Elasticsearch 的企业级检索,其底层都建立在经典的\textbf{倒排索引}和\textbf{相关性排序}机制之上。本文将通过 Python 实现一个支持中文检索的简易搜索引擎,帮助开发者理解其核心原理与技术细节。\par
\chapter{技术原理}
全文搜索引擎的核心在于\textbf{倒排索引}这一数据结构。与传统书籍目录的「页码→内容」映射不同,倒排索引建立的是「关键词→文档集合」的反向映射。例如对于文档集合:\par
\begin{lstlisting}
文档 1:"搜索引擎原理与实践"
文档 2:"Python 实现搜索引擎"
\end{lstlisting}
构建的倒排索引将呈现为:\par
\begin{lstlisting}
"搜索引擎" → {文档 1:1, 文档 2:1}
"Python" → {文档 2:1}
\end{lstlisting}
相关性排序通常采用 TF-IDF 算法,其公式为:\par
$$ \text{TF-IDF} = \text{TF}(t,d) \times \text{IDF}(t) $$\par
其中 $\text{TF}(t,d)$ 表示词项 $t$ 在文档 $d$ 中的出现频率,$\text{IDF}(t) = \log(\frac{N}{n_t})$ 表示逆文档频率($N$ 为总文档数,$n_t$ 为包含词项 $t$ 的文档数)。该算法同时考虑词项的局部重要性和全局区分度。\par
\chapter{环境准备}
我们选择 Python 作为开发语言,因其丰富的文本处理库和简洁的语法特性。中文分词采用 \verb!jieba! 库,其具备 98\%{} 以上的分词准确率和自定义词典支持。通过以下命令安装依赖:\par
\begin{lstlisting}[language=bash]
pip install jieba
\end{lstlisting}
\chapter{核心实现步骤}
\section{文档预处理与倒排索引构建}
搜索引擎首先需要将原始文本转化为结构化的索引数据。以下代码实现文档添加和索引构建:\par
\begin{lstlisting}[language=python]
from collections import defaultdict
import jieba

class SimpleSearchEngine:
    def __init__(self):
        self.inverted_index = defaultdict(dict)  # 倒排索引结构
        self.documents = []  # 文档原始内容存储
        self.stop_words = set(["的", "是", "在"])  # 示例停用词表

    def add_document(self, doc_id, text):
        # 中文分词与清洗
        words = jieba.lcut(text)
        words = [word.lower() for word in words 
                if word not in self.stop_words and len(word) > 1]
        
        # 更新倒排索引
        for word in words:
            if doc_id not in self.inverted_index[word]:
                self.inverted_index[word][doc_id] = 0
            self.inverted_index[word][doc_id] += 1
        
        self.documents.append({"id": doc_id, "content": text})
\end{lstlisting}
代码解读:\par
\begin{enumerate}
\item \verb!defaultdict(dict)! 创建嵌套字典,外层键为词项,内层键为文档 ID
\item \verb!jieba.lcut! 执行中文分词,\verb!lower()! 统一为小写形式
\item 停用词过滤移除「的」等无意义词汇,提升索引质量
\end{enumerate}
\section{搜索逻辑实现}
当用户输入查询时,系统需要完成分词、索引查找和结果合并:\par
\begin{lstlisting}[language=python]
def search(self, query, page=1, per_page=10):
    query_terms = jieba.lcut(query)
    matched_docs = set()
    
    # 收集所有包含查询词的文档
    for term in query_terms:
        if term in self.inverted_index:
            matched_docs.update(self.inverted_index[term].keys())
    
    # 计算相关性排序
    ranked = self.rank_documents(query_terms)
    
    # 分页处理
    start = (page - 1) * per_page
    return ranked[start:start + per_page]
\end{lstlisting}
该实现采用 OR 逻辑合并结果,即返回包含任意查询词的文档。实际工业级系统通常支持更复杂的布尔运算。\par
\section{相关性排序优化}
基于 TF-IDF 的排序算法实现如下:\par
\begin{lstlisting}[language=python]
import math

def rank_documents(self, query_terms):
    scores = defaultdict(float)
    total_docs = len(self.documents)
    
    for term in query_terms:
        if term not in self.inverted_index:
            continue
        
        # 计算 IDF 值
        doc_count = len(self.inverted_index[term])
        idf = math.log(total_docs / (doc_count + 1))
        
        # 累加 TF-IDF 分数
        for doc_id, tf in self.inverted_index[term].items():
            scores[doc_id] += tf * idf
    
    # 按分数降序排列
    return sorted(scores.items(), key=lambda x: x[1], reverse=True)
\end{lstlisting}
代码关键点:\par
\begin{enumerate}
\item \verb!math.log! 计算自然对数,避免零除错误加入 +1 平滑
\item 分数累加策略使包含多个查询词的文档获得更高排名
\item 排序时间复杂度为 $O(n \log n)$,适用于中小规模数据集
\end{enumerate}
\section{分页与结果展示}
分页功能通过列表切片实现,以下代码演示结果格式化输出:\par
\begin{lstlisting}[language=python]
def format_results(self, ranked_docs):
    results = []
    for doc_id, score in ranked_docs:
        content = self.documents[doc_id]["content"]
        # 截取摘要(前 100 字符)
        snippet = content[:100] + "..." if len(content) > 100 else content
        results.append({
            "id": doc_id,
            "score": round(score, 2),
            "snippet": snippet
        })
    return results
\end{lstlisting}
\chapter{优化与扩展方向}
在基础版本之上,可通过以下方式提升系统性能与功能:\par
\begin{enumerate}
\item \textbf{前缀匹配优化}:引入 Trie 树实现自动补全,将时间复杂度从 $O(n)$ 降至 $O(k)$($k$ 为查询词长度)
\item \textbf{缓存机制}:使用 LRU 缓存存储高频查询结果,降低重复计算开销
\item \textbf{短语搜索}:通过 Bigram 索引记录词语位置信息,支持精确短语匹配
\item \textbf{拼写纠错}:基于编辑距离(Levenshtein Distance)实现查询词建议
\end{enumerate}
例如拼写纠错的核心逻辑可表示为:\par
$$ \text{编辑距离}(s, t) = \min \begin{cases} \text{删除操作} & \text{插入操作} \\ \text{替换操作} & \text{空字符串长度} \end{cases} $$\par
本文实现的搜索引擎虽然省略了分布式、实时更新等复杂特性,但完整呈现了倒排索引构建、TF-IDF 排序等核心机制。读者可通过扩展停用词表、引入 BM25 算法、增加持久化存储等方式继续完善系统。深入学习建议参考《信息检索导论》和 Lucene 源码,探索 PageRank 等更复杂的排序模型。\par
