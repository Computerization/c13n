\title{"深入理解并实现基本的双端队列(Deque)数据结构"}
\author{"叶家炜"}
\date{"Jul 01, 2025"}
\maketitle
双端队列(Deque,全称 Double-Ended Queue)是一种支持在两端高效进行插入和删除操作的线性数据结构。与传统队列严格的 FIFO(先进先出)规则和栈的 LIFO(后进先出)规则不同,Deque 融合了两者的特性,允许开发者根据需求自由选择操作端。这种灵活性使其成为解决特定问题的利器。\par
为什么需要 Deque?在实际开发中,诸多场景需要两端操作能力。例如实现撤销操作历史记录时,新操作从前端加入而旧操作从后端移除;滑动窗口算法中需要同时维护窗口两端的数据;工作窃取算法和多线程任务调度也依赖双端操作特性。Deque 的核心操作包括 \texttt{addFront}/\texttt{addRear} 插入、\texttt{removeFront}/\texttt{removeRear} 删除以及 \texttt{peekFront}/\texttt{peekRear} 查看操作,这些构成了其基本能力集。\par
\chapter{双端队列的抽象行为与操作}
理解 Deque 需要明确其操作定义与边界条件。前端插入 \texttt{addFront(item)} 和后端插入 \texttt{addRear(item)} 在队列满时需扩容;删除操作 \texttt{removeFront()} 和 \texttt{removeRear()} 在空队列时报错;辅助方法 \texttt{isEmpty()} 判断队列空状态,\texttt{size()} 返回元素数量。这些操作共同定义了 Deque 的抽象行为。\par
可视化理解操作流程:假设初始为空队列,执行 \texttt{addFront(A)} 后队列为「A」;接着 \texttt{addRear(B)} 形成「A ←→ B」结构;执行 \texttt{removeFront()} 移除 A 剩下「B」;最后 \texttt{removeRear()} 移除 B 回归空队列。这种动态过程清晰展示了 Deque 的双端操作特性。\par
\chapter{实现方案:双向链表与循环数组}
\section{双向链表实现方案}
双向链表方案通过节点间的双向指针实现高效端操作。节点类设计包含数据域和前后指针:\par
\begin{lstlisting}[language=python]
class Node:
    def __init__(self, data):
        self.data = data
        self.next = None
        self.prev = None
\end{lstlisting}
队列主体维护头尾指针和大小计数器:\par
\begin{lstlisting}[language=python]
class LinkedListDeque:
    def __init__(self):
        self.front = None  # 头指针指向首节点
        self.rear = None   # 尾指针指向末节点
        self._size = 0
\end{lstlisting}
\texttt{addFront} 操作创建新节点并更新头指针:新节点 next 指向原头节点,原头节点 prev 指向新节点。时间复杂度稳定为 $O(1)$,无扩容开销。优势在于动态扩容灵活,代价是每个节点需额外存储两个指针,空间开销为 $O(n) + 2 \times n \times ptr_{size}$。\par
\section{循环数组实现方案}
循环数组方案使用固定容量数组,通过模运算实现逻辑循环:\par
\begin{lstlisting}[language=python]
class ArrayDeque:
    def __init__(self, capacity=10):
        self.capacity = max(1, capacity)
        self.items = [None] * self.capacity
        self.front = 0  # 指向队首元素索引
        self.rear = 0   # 指向队尾后第一个空位索引
        self.size = 0
\end{lstlisting}
核心在于下标的循环计算:\texttt{index = (current + offset) \%{} capacity}。队列满判断依据为 \texttt{(rear + 1) \%{} capacity == front}。均摊时间复杂度为 $O(1)$,但扩容时需 $O(n)$ 数据迁移。优势是内存连续访问高效,缺陷是扩容需数据搬移。\par
\chapter{代码实现:循环数组详解}
以下为循环数组实现的完整代码,含详细注释:\par
\begin{lstlisting}[language=python]
class ArrayDeque:
    def __init__(self, capacity=10):
        self.capacity = max(1, capacity)  # 确保最小容量为 1
        self.items = [None] * self.capacity
        self.front = 0  # 指向第一个有效元素
        self.rear = 0    # 指向下一个插入位置
        self.size = 0    # 当前元素数量

    def _resize(self, new_cap):
        """扩容迁移数据,保持元素物理顺序"""
        new_items = [None] * new_cap
        # 按逻辑顺序复制元素:从 front 开始连续取 size 个
        for i in range(self.size):
            new_items[i] = self.items[(self.front + i) % self.capacity]
        self.items = new_items
        self.front = 0      # 重置 front 到新数组首
        self.rear = self.size  # rear 指向最后一个元素后
        self.capacity = new_cap

    def addFront(self, item):
        """前端插入:front 逆时针移动"""
        if self.size == self.capacity:
            self._resize(2 * self.capacity)  # 容量翻倍
        # 计算新 front 位置(循环左移)
        self.front = (self.front - 1) % self.capacity
        self.items[self.front] = item
        self.size += 1

    def addRear(self, item):
        """后端插入:直接写入 rear 位置"""
        if self.size == self.capacity:
            self._resize(2 * self.capacity)
        self.items[self.rear] = item
        self.rear = (self.rear + 1) % self.capacity
        self.size += 1

    def removeFront(self):
        if self.isEmpty():
            raise Exception("Deque is empty")
        item = self.items[self.front]
        self.front = (self.front + 1) % self.capacity  # 顺时针移动
        self.size -= 1
        return item

    def removeRear(self):
        if self.isEmpty():
            raise Exception("Deque is empty")
        # rear 指向空位,需先回退到末元素
        self.rear = (self.rear - 1) % self.capacity
        item = self.items[self.rear]
        self.size -= 1
        return item
\end{lstlisting}
扩容函数 \texttt{\_{}resize} 通过遍历原数组,按逻辑顺序(从 \texttt{front} 开始)复制元素到新数组,确保数据连续性。前端插入时 \texttt{front} 逆时针移动(索引减一),利用模运算处理越界;后端插入直接写入 \texttt{rear} 位置并顺时针移动。删除操作需特别注意 \texttt{removeRear} 时 \texttt{rear} 指向空位,需先回退获取末元素。\par
\chapter{复杂度与性能对比}
两种实现方案的时间复杂度对比显著:\par
\begin{table}[H]
\centering
\begin{tabular}{|l|l|l|}
\hline
\textbf{操作} & \textbf{双向链表} & \textbf{循环数组(均摊)} \\
\hline
\texttt{addFront} & $O(1)$ & $O(1)$ \\
\hline
\texttt{addRear} & $O(1)$ & $O(1)$ \\
\hline
\texttt{removeFront} & $O(1)$ & $O(1)$ \\
\hline
\texttt{removeRear} & $O(1)$ & $O(1)$ \\
\hline
\end{tabular}
\end{table}
空间开销方面:双向链表需 $O(n)$ 基础空间加上 $2 \times n \times ptr_{size}$ 指针开销;循环数组仅需 $O(n)$ 连续空间但可能包含空闲位。选择依据明确:频繁动态伸缩场景用双向链表,已知最大容量时循环数组更优。\par
\chapter{应用场景实战}
\section{滑动窗口最大值(LeetCode 239)}
Deque 在此算法中维护单调递减序列:\par
\begin{lstlisting}[language=python]
deque = ArrayDeque()
result = []
for i, num in enumerate(nums):
    # 清除小于当前值的尾部元素
    while not deque.isEmpty() and num > nums[deque.peekRear()]:
        deque.removeRear()
    deque.addRear(i)  # 存入当前索引
    # 移除移出窗口的头部元素
    if deque.peekFront() == i - k:
        deque.removeFront()
    # 记录窗口最大值
    if i >= k - 1:
        result.append(nums[deque.peekFront()])
\end{lstlisting}
Deque 头部始终存储当前窗口最大值索引。当新元素 $nums_i$ 加入时,循环移除尾部小于 $nums_i$ 的元素,确保队列单调递减。同时检测并移除超出窗口的头部元素。该实现时间复杂度优化至 $O(n)$。\par
\section{多层级撤销操作}
在支持多级撤销的编辑器中,Deque 可高效管理操作历史:\par
\begin{lstlisting}[language=python]
class UndoManager:
    def __init__(self, max_history=100):
        self.history = ArrayDeque(max_history)
        self.redo_stack = []

    def execute(self, command):
        command.execute()
        self.history.addFront(command)  # 新操作前端插入
        self.redo_stack.clear()

    def undo(self):
        if not self.history.isEmpty():
            cmd = self.history.removeFront()  # 移除最近操作
            cmd.undo()
            self.redo_stack.append(cmd)  # 存入重做栈
\end{lstlisting}
新操作从 Deque 前端插入,撤销时移除前端操作。当历史记录达到容量上限时,最旧操作自动从后端移除。这种设计完美平衡了空间效率和操作时效性。\par
双端队列的核心价值在于双端操作的高效性与栈/队列特性的统一抽象。实现选择需权衡场景:小规模动态数据适用双向链表;大规模预知容量数据优选循环数组。延伸思考包括线程安全实现方案(如加锁或原子操作)和循环数组内存碎片优化策略(如间隙压缩算法)。\par
测试用例验证实现正确性:\par
\begin{lstlisting}[language=python]
def test_ArrayDeque():
    dq = ArrayDeque(3)
    dq.addRear(2)        # 状态 : [2]
    dq.addFront(1)       # 状态 : [1, 2]
    dq.addRear(3)        # 状态 : [1, 2, 3] → 触发扩容
    assert dq.size == 3
    assert dq.removeFront() == 1  # 状态 : [2, 3]
    assert dq.removeRear() == 3   # 状态 : [2]
    assert not dq.isEmpty()
\end{lstlisting}
该用例覆盖基础操作、边界扩容和状态转换,确保实现符合预期。掌握 Deque 将显著提升开发者解决复杂问题的能力。\par
