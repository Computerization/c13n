\title{"深入理解并实现基本的基数排序(Radix Sort)算法"}
\author{"叶家炜"}
\date{"Aug 04, 2025"}
\maketitle
排序算法在计算机科学中占据基础地位,广泛应用于数据处理、数据库索引、搜索算法等多个领域。常见的排序算法可分为比较排序(如快速排序、归并排序)和非比较排序两大类。基数排序作为非比较排序的代表,以其线性时间复杂度的特性脱颖而出,特别适用于整数或字符串等可分解键值的数据类型。本文旨在透彻解析基数排序的原理,通过手把手实现代码加深理解,并分析其性能与适用边界,帮助读者掌握这一高效算法。\par
\chapter{基数排序的核心思想}
基数排序的核心在于「基数」的概念,即键值的进制基数,如十进制中基数为 10。排序过程通过按位进行,从最低位到最高位(LSD 方式),在多轮「分桶-收集」操作中完成。这类似于整理扑克牌时,先按花色分桶,再按点数排序。关键特性是每轮排序必须保持稳定性,即相同键值的元素在排序后保持原顺序,这对算法的正确性至关重要。稳定性确保在后续高位排序时,低位排序的结果不被破坏。\par
\chapter{算法步骤详解}
基数排序的算法步骤包括预处理和核心循环。预处理阶段需要确定数组中最大数字的位数 $d$,这决定了排序轮数。核心循环中,每轮针对一个位进行分桶与收集操作。具体步骤为:首先创建 10 个桶(对应数字 0 到 9);然后按当前位数字将元素分配到相应桶中,确保分配过程稳定;接着按桶顺序(从 0 到 9)收集所有元素回原数组;最后更新当前位向高位移动,重复此过程直至最高位。以数组 $[170, 45, 75, 90, 802, 24, 2, 66]$ 为例,第一轮按个位分桶:桶 0 包含 170 和 90,桶 2 包含 802 和 2,桶 4 包含 24,桶 5 包含 45 和 75,桶 6 包含 66;收集后数组为 $[170, 90, 802, 2, 24, 45, 75, 66]$;第二轮按十位分桶:桶 0 包含 802 和 2,桶 6 包含 66,桶 7 包含 170 和 75,桶 9 包含 90;收集后数组为 $[802, 2, 24, 66, 170, 75, 90, 45]$;第三轮按百位分桶后收集,最终得到有序数组 $[2, 24, 45, 66, 75, 90, 170, 802]$。\par
\chapter{时间复杂度与空间复杂度分析}
基数排序的时间复杂度为 $O(d \cdot (n + k))$,其中 $d$ 是最大位数, $n$ 是元素数量, $k$ 是基数(桶的数量)。与比较排序如快速排序的 $O(n \log n)$ 相比,当 $d$ 较小且 $k$ 不大时,基数排序效率更高,尤其在数据规模大但位数少的场景。空间复杂度为 $O(n + k)$,主要来自桶的额外存储。稳定性是算法成立的前提,因为每轮排序必须是稳定的,以保证高位排序时低位顺序不被破坏;如果某轮排序不稳定,整体结果可能出错。\par
\chapter{基数排序的局限性}
尽管高效,基数排序有显著局限性。它主要适用于整数、定长字符串(需补位)或前缀可比较的数据类型,不直接处理浮点数或可变长数据(需额外转换)。当基数 $k$ 较大时,如处理 Unicode 字符串,空间开销显著增加。此外,如果位数 $d$ 接近元素数 $n$,算法可能退化为 $O(n^2)$ 效率,例如在大范围稀疏数据中。\par
\chapter{代码实现(Python 示例)}
以下是用 Python 实现的基数排序代码:\par
\begin{lstlisting}[language=python]
def radix_sort(arr):
    # 1. 计算最大位数
    max_digits = len(str(max(arr)))
    
    # 2. LSD 排序循环
    for digit in range(max_digits):
        # 创建 10 个桶
        buckets = [[] for _ in range(10)]
        
        # 按当前位分配元素
        for num in arr:
            current_digit = (num // (10 ** digit)) % 10
            buckets[current_digit].append(num)
        
        # 收集元素(保持桶内顺序)
        arr = [num for bucket in buckets for num in bucket]
    
    return arr

# 测试
arr = [170, 45, 75, 90, 802, 24, 2, 66]
print("排序前 :", arr)
print("排序后 :", radix_sort(arr))
\end{lstlisting}
代码解读:首先,在预处理阶段,\texttt{max\_{}digits = len(str(max(arr)))} 计算数组中最大数字的位数,例如最大数 802 的位数为 3。在 LSD 循环中,变量 \texttt{digit} 表示当前处理的位索引(从 0 开始,0 为个位)。对于每个数字 \texttt{num},\texttt{current\_{}digit = (num // (10 ** digit)) \%{} 10} 提取当前位数字:例如当 \texttt{digit=0} 时,170 的个位为 $(170 // 10^0) \% 10 = 170 \% 10 = 0$。元素被分配到 \texttt{buckets} 列表中,桶使用列表的列表实现,确保稳定性(相同当前位数字的元素保持原顺序)。收集操作 \texttt{arr = [num for bucket in buckets for num in bucket]} 通过列表推导式将所有桶中的元素扁平化回数组,保持桶内顺序。测试部分输出排序前后数组,验证算法正确性。\par
\chapter{变体与优化}
基数排序有多个变体与优化方向。MSD(最高位优先)基数排序采用递归方式,先按最高位分桶,再对每个桶递归排序,适合字符串处理。在桶的实现上,可用链表代替动态数组以减少内存分配开销;尝试原地排序虽复杂但可能节省空间,但需牺牲稳定性。对于负数处理,可分离正负数分别排序,或通过添加偏移量(如加 1000)将负数转为正数处理后再排序,最后还原符号。\par
\chapter{实际应用场景}
基数排序在实际中常用于大范围整数排序,如数据库索引构建或大规模 ID 排序,其中数据量大但位数有限。它也适用于定长字符串的字典序排序,例如车牌号或 ISBN 号的快速处理。此外,基数排序可扩展至混合键值排序场景,如先按日期(高位)再按 ID(低位)的多级排序,充分利用其稳定性优势。\par
基数排序的核心优势在于其线性时间复杂度 $O(d \cdot (n + k))$,突破比较排序的下限 $O(n \log n)$。使用时需满足键值可分解、排序过程稳定且空间充足等前提。学习基数排序不仅掌握一种高效算法,更体现了非比较排序的设计思想和空间换时间的经典权衡,为处理特定数据类型提供优化方案。\par
