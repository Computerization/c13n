\title{PyTorch 在移动和边缘设备上的部署}
\author{叶家炜}
\date{Dec 23, 2025}
\maketitle
边缘计算和移动 AI 的兴起源于对低延迟、隐私保护以及离线能力的迫切需求。在传统的云端 AI 部署中,数据传输带来的延迟往往难以满足实时应用场景,而将模型直接运行在设备端则能有效规避这些问题。同时,用户隐私数据无需上传云端,进一步提升了安全性。PyTorch 作为 AI 开发领域的热门框架,以其动态图和灵活性深受开发者青睐,但其在边缘部署上面临模型体积庞大、计算资源受限以及跨平台兼容性等挑战。本文旨在提供从模型训练到边缘部署的全流程指南,针对初学者和中级开发者,分享实用工具和最佳实践。读者需具备基本的 PyTorch 知识以及 Android 或 iOS 移动开发基础。\par
\chapter{2. PyTorch 边缘部署生态概述}
PyTorch 的边缘部署生态由一系列核心工具栈构成,这些工具共同支撑从模型导出到运行的全链路。TorchScript 是 PyTorch 原生模型序列化格式,支持 Android、iOS 和 Linux 平台,通过它可以将动态图转换为静态图以提升执行效率。PyTorch Mobile 则提供专为移动端优化的运行时,直接集成到 Android 和 iOS 应用中。ExecuTorch 作为 PyTorch 2.0 之后的下一代运行时,针对嵌入式设备设计,具有更小的二进制体积和更低的内存占用。此外,ONNX 格式允许跨框架导出,并搭配相应运行时支持多平台部署,而 TorchServe 及其移动变体则适用于服务器和边缘的服务化场景。整个部署流程可概括为训练模型、进行优化、导出格式、集成到应用、运行推理并持续调优性能,这一流程确保了从开发到生产的顺畅过渡。\par
\chapter{3. 模型准备与优化}
模型优化是边缘部署的基础,其中量化技术尤为关键。通过将浮点权重转换为 INT8 或 FP16 格式,可以显著减少模型大小和计算量。PyTorch 的 \texttt{torch.quantization} 模块支持动态和静态量化两种模式。以动态量化为例,它在推理时实时量化激活值,而权重预先量化。这种方法简单易用,适用于快速原型验证。以下是动态量化的示例代码:\par
\begin{lstlisting}[language=python]
import torch
import torch.quantization
model = torch.hub.load('pytorch/vision', 'resnet18', pretrained=True)
model.eval()
model.qconfig = torch.quantization.get_default_qconfig('fbgemm')
torch.quantization.prepare(model, inplace=True)
# 校准数据模拟
calib_data = torch.randn(10, 3, 224, 224)
for data in calib_data:
    model(data)
quantized_model = torch.quantization.convert(model, inplace=False)
\end{lstlisting}
这段代码首先加载预训练的 ResNet-18 模型,并设置为评估模式。然后配置量化方案,使用 \texttt{fbgemm} 后端适合 x86 架构。\texttt{prepare} 函数插入量化节点,之后通过校准数据(如随机生成的图像张量)收集统计信息,最终 \texttt{convert} 函数完成量化转换。量化后模型大小可减少 4 倍左右,但需注意精度损失,可通过 Top-1 准确率评估。\par
剪枝和知识蒸馏进一步优化模型,前者移除冗余权重,后者用大模型指导小模型训练。对于 TorchScript 导出,有两种主要方法:\texttt{torch.jit.trace} 通过示例输入追踪计算图,适合无控制流的模型;\texttt{torch.jit.script} 则编译 Python 代码,支持 if-else 等逻辑,但需注解复杂函数。选择取决于模型特性。以 \texttt{torch.jit.trace} 导出 CNN 模型为例:\par
\begin{lstlisting}[language=python]
model = MyCNN()
model.eval()
example_input = torch.randn(1, 3, 224, 224)
traced_model = torch.jit.trace(model, example_input)
traced_model.save("model.pt")
\end{lstlisting}
这里定义自定义 CNN 模型,传入示例输入进行追踪,生成静态图并保存为 \texttt{.pt} 文件。常见问题包括控制流不支持,可用 \texttt{script} 解决;动态形状则需固定输入尺寸或使用 padding 处理。\par
PyTorch 2.1 引入的 ExecuTorch 进一步提升边缘性能,其优势在于支持更多算子、更小二进制和低内存占用。导出流程使用 \texttt{torch.export}:\par
\begin{lstlisting}[language=python]
import torch.export
model = MyModel()
example_args = (torch.randn(1, 3, 224, 224),)
exported_program = torch.export.export(model, example_args)
exported_program.save("model.ep")
\end{lstlisting}
此代码捕获模型与输入的联合表示,生成 \texttt{.ep} 文件,支持后续 AOT 编译,适用于资源极度受限的设备。\par
\chapter{4. 平台特定部署指南}
\section{4.1 Android 部署(PyTorch Mobile)}
在 Android 上部署需先搭建环境,包括 Android Studio、NDK,并通过 Gradle 添加 PyTorch Mobile AAR 依赖。集成步骤从加载 TorchScript 模型开始,使用 \texttt{Module.load} 从 assets 读取模型文件。随后进行输入预处理,将 Bitmap 转换为 Tensor,并执行推理。完整图像分类示例代码如下:\par
\begin{lstlisting}[language=java]
Module module = Module.load(assetFilePath(this, "model.pt"));
Tensor inputTensor = ImageUtils.bitmapToFloat32Tensor(bitmap, 224, 224, 3);
IValue inputs = IValue.from(inputTensor);
Tensor outputTensor = module.forward(IValue.from(inputs)).toTensor();
float[] scores = outputTensor.getDataAsFloatArray();
\end{lstlisting}
这段 Java 代码首先加载模型,然后利用工具函数将图像转换为 normalized Float32 Tensor(尺寸 224×224,通道 3)。\texttt{forward} 方法接收 IValue 包装的输入,返回输出 Tensor,最后提取概率分数进行分类(如 argmax 取 Top-1)。为优化性能,可启用 NNAPI 委托加速 GPU/NPU,或通过 JNI 最小化 Java-Kotlin 桥接开销。\par
\section{4.2 iOS 部署(PyTorch Mobile)}
iOS 部署通过 CocoaPods 集成 \texttt{LibTorch-Core},在 Xcode 中配置后即可使用。通过 \texttt{MobileModule.loadModel} 加载模型,并处理输入 Tensor。Swift 示例代码如下:\par
\begin{lstlisting}[language=swift]
let module = try MobileModule.loadModel(modelPath: modelPath)
let inputTensor = MobileTensor.fromBlob(blob: inputBlob, shape: [1, 3, 224, 224])
let output = try module.forward(input: [MobileArgument(inputTensor)]).get<MobileTensor>(0)
let scores = output.multiDimArray()!.data.floats
\end{lstlisting}
此代码加载模型,从 Blob 数据创建输入 Tensor(需预先从 UIImage 转换),调用 \texttt{forward} 执行推理,并从输出中提取浮点数组。性能提升可通过转换为 CoreML 格式实现:使用 \texttt{coremltools} 将 TorchScript 导出为 \texttt{.mlmodel},集成 Metal 或 ANE 加速,推理速度可提升 2-3 倍。\par
\section{4.3 嵌入式设备(Raspberry Pi / Microcontrollers)}
对于 Raspberry Pi 等 Linux ARM 设备,ExecuTorch 通过 \texttt{pip install executorch} 安装,支持语音识别等任务。微控制器如 STM32 或 ESP32 受限于内存,仅支持核心算子,通过 XLA 后端编译生成的 C++ 代码运行。\par
\chapter{5. 高级优化与性能调优}
硬件加速是性能关键。在 Android 上,PyTorch Mobile 通过 NNAPI 委托调用 GPU 或 NPU;iOS 使用 CoreML 集成 ANE 和 Metal;边缘 NPU 如 Qualcomm 的则依赖 ExecuTorch 后端。基准测试采用 TensorFlow Lite Benchmark 工具结合 PyTorch Profiler,关注延迟、内存、功耗和 Top-1 准确率等指标。常见瓶颈包括内存爆炸,可通过设置 Batch=1 和静态形状解决;冷启动慢则用 AOT 编译预热。\par
\chapter{6. 实际案例与最佳实践}
在移动图像分类案例中,将 MobileNetV3 导出为 TorchScript 并部署到 Android,量化后模型大小降至 10MB,推理延迟 20ms,对比浮点版精度损失小于 1\%{}。边缘实时目标检测则将 YOLOv5 转为 ONNX,再用 ExecuTorch 在 Jetson Nano 上运行,达到 30 FPS。最佳实践包括控制模型大小低于 50MB、推理延迟低于 30ms,使用 Git LFS 版本控制模型,并集成 Torch Hub 到 CI/CD 管道。\par
\chapter{7. 挑战与未来展望}
当前挑战包括算子支持不全、动态形状处理困难以及跨平台一致性问题。未来,PyTorch 2.x 通过 TorchDynamo 和 ExecuTorch 扩展生态,FBGEMM/TVM 集成深化硬件支持,联邦学习也将释放边缘潜力。\par
\chapter{8. 结论与资源}
PyTorch 边缘部署提供从 TorchScript 到 ExecuTorch 的完整路径,开发者可据需选择。立即实践官方 GitHub 示例仓库。进一步资源包括 pytorch.org/mobile 文档、pytorch.org/executorch 页面、github.com/pytorch/mobile 示例以及 PyTorch Forums 社区。\par
\textbf{附录}:完整代码仓库见 github.com/pytorch/android-demo。FAQ 示例:量化精度下降时,使用 QAT(量化感知训练)在训练中模拟量化误差,或增加校准数据集大小。\par
