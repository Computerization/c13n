\title{基本的图(Graph)数据结构}
\author{杨其臻}
\date{Jul 23, 2025}
\maketitle
图作为一种强大的数据结构,在现实世界中无处不在。社交网络中的人际关系、地图导航中的路径规划、推荐系统中的用户行为建模,都依赖于图的抽象能力。在计算机科学领域,图常被视为数据结构的“天花板”之一,因为它能高效处理复杂的关联关系。本文的目标是系统讲解图的核心概念,包括图的定义、分类和存储方式,并手把手实现两种基础存储结构:邻接矩阵和邻接表。我们还将实现关键算法如深度优先搜索(DFS)和广度优先搜索(BFS),分析其适用场景,并提供完整的 Python 代码示例。通过这些内容,读者将构建对图的系统认知,为实际项目应用奠定基础。\par
\chapter{图的基础理论}
图在数学和编程中的核心定义是一致的:它由顶点(Vertex 或 Node)和边(Edge)组成。抽象表示为 $G = (V, E)$,其中 $V$ 是顶点集合,$E$ 是边集合。例如,在社交网络中,用户是顶点,好友关系是边;在地图导航中,地点是顶点,道路是边。图的分类基于多个维度:有向图与无向图的区别在于边是否有方向性(如网页链接是有向的,好友关系通常是无向的);加权图与无权图则涉及边是否携带权重(如导航距离是加权,好友关系是无权)。关键概念包括连通图(所有顶点相互可达)、环(路径形成闭环)、度(顶点的连接数,入度/出度用于有向图),以及稀疏图与稠密图(边数远少于或接近顶点数的平方),这些概念直接影响存储结构的选择。\par
\chapter{图的存储结构:如何用代码表示图?}
图的存储结构决定了算法的效率和适用性。邻接矩阵使用二维数组表示顶点间的边关系。例如,\texttt{matrix[i][j]} 存储顶点 \texttt{i} 到 \texttt{j} 的边权重。以下 Python 实现展示其核心逻辑:\par
\begin{lstlisting}[language=python]
class GraphMatrix:
    def __init__(self, num_vertices):
        self.matrix = [[0] * num_vertices for _ in range(num_vertices)]
    
    def add_edge(self, v1, v2, weight=1):
        self.matrix[v1][v2] = weight
        # 若是无向图,需对称添加:self.matrix[v2][v1] = weight
\end{lstlisting}
在解读这段代码时,\texttt{\_{}\_{}init\_{}\_{}} 方法初始化一个大小为 $V \times V$ 的矩阵($V$ 是顶点数),所有元素初始化为 0 表示无边。\texttt{add\_{}edge} 方法添加边:参数 \texttt{v1} 和 \texttt{v2} 指定起点和终点,\texttt{weight} 默认为 1(无权图)。关键点在于注释部分:对于无向图,必须对称设置 \texttt{matrix[v2][v1]},以确保双向关系。邻接矩阵的空间复杂度为 $O(V^2)$,适合稠密图;优点是快速($O(1)$)判断边存在性;缺点是空间浪费于稀疏图,且添加/删除顶点需重建矩阵,成本较高。\par
邻接表则更高效地处理稀疏图,使用数组加链表结构,每个顶点维护其邻接点列表。以下实现展示核心逻辑:\par
\begin{lstlisting}[language=python]
class GraphList:
    def __init__(self, num_vertices):
        self.adj_list = [[] for _ in range(num_vertices)]
    
    def add_edge(self, v1, v2, weight=1):
        self.adj_list[v1].append((v2, weight))  # 存储目标节点和权重
        # 无向图需反向添加:self.adj_list[v2].append((v1, weight))
\end{lstlisting}
代码解读:\texttt{\_{}\_{}init\_{}\_{}} 创建大小为 $V$ 的列表,每个元素是空列表。\texttt{add\_{}edge} 将目标顶点 \texttt{v2} 和权重添加到 \texttt{v1} 的邻接表中;注释强调无向图需反向添加以保持对称。邻接表空间复杂度为 $O(V + E)$($E$ 是边数),查询邻接点仅需 $O(\text{degree}(V))$(顶点的度数),但无法 $O(1)$ 判断任意边存在性。总结对比:邻接矩阵在稠密图中优势明显,而邻接表在稀疏图中更节省空间;添加顶点时,邻接表开销低;遍历邻接点时,邻接表基于度数更高效。\par
\chapter{图的遍历:探索图的基础算法}
遍历算法是图分析的基石。深度优先搜索(DFS)采用“一条路走到底”的策略,使用栈实现回溯机制。它适用于拓扑排序、连通分量检测等场景。以下是基于邻接表的 Python 实现:\par
\begin{lstlisting}[language=python]
def dfs(graph, start):
    visited = set()
    stack = [start]
    while stack:
        vertex = stack.pop()
        if vertex not in visited:
            visited.add(vertex)
            for neighbor, _ in graph.adj_list[vertex]:
                if neighbor not in visited:
                    stack.append(neighbor)
    return visited
\end{lstlisting}
代码解读:\texttt{visited} 集合记录已访问顶点,避免重复。\texttt{stack} 初始化为起始点;循环中弹出顶点,若未访问则标记,并遍历其邻接点(\texttt{graph.adj\_{}list[vertex]} 返回邻接列表)。关键点在于 \texttt{stack.append(neighbor)} 将未访问邻居压栈,确保深度优先。时间复杂度为 $O(V + E)$,空间复杂度 $O(V)$。\par
广度优先搜索(BFS)采用“层层推进”策略,使用队列实现最短路径基础。它适用于无权图最短路径或社交网络扩展。实现如下:\par
\begin{lstlisting}[language=python]
from collections import deque
def bfs(graph, start):
    visited = set([start])
    queue = deque([start])
    while queue:
        vertex = queue.popleft()
        for neighbor, _ in graph.adj_list[vertex]:
            if neighbor not in visited:
                visited.add(neighbor)
                queue.append(neighbor)
    return visited
\end{lstlisting}
解读:\texttt{visited} 和队列初始包含起始点。循环中,\texttt{queue.popleft()} 取出顶点,遍历其邻接点;未访问邻居被标记并加入队列尾部(\texttt{queue.append(neighbor)})。这保证了层级遍历,时间复杂度同样 $O(V + E)$。BFS 与 DFS 的核心差异在于遍历顺序:BFS 找到最短路径(无权图),DFS 更易检测环或深度结构。\par
\chapter{图的应用场景与进阶算法预告}
图在工程中广泛应用。路径规划依赖 Dijkstra 算法(加权最短路径)或 A*算法(启发式搜索);网络分析中,PageRank 算法通过链接结构评估网页重要性;社交网络使用 BFS 扩展好友推荐(如三级好友关系)。这些应用凸显图的建模能力。进阶方向包括最小生成树算法(如 Prim 或 Kruskal 用于网络优化)、拓扑排序(处理任务依赖)、以及强连通分量算法(如 Tarjan 算法分析子图结构)。掌握这些算法能解决复杂问题如芯片布线或数据流分析。\par
本文系统回顾了图的关键点:邻接矩阵和邻接表作为存储结构,需根据图类型(稠密或稀疏)选择;DFS 和 BFS 是基础遍历算法,前者适合深度探索,后者用于最短路径。图的重要性在于其普适性:从微观的芯片设计到宏观的宇宙网络建模,都能高效处理关联关系。建议下一步在项目中实践,如使用 NetworkX 图库或挑战 LeetCode 题目(如课程调度问题),以深化理解。\par
