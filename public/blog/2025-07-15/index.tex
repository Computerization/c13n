\title{"深入理解并查集(Disjoint Set Union)"}
\author{"叶家炜"}
\date{"Jul 15, 2025"}
\maketitle
在计算机科学中,动态连通性问题是一个经典挑战。想象一个社交网络场景:用户 A 和 B 成为好友后,我们需要快速判断任意两个用户是否属于同一个朋友圈。传统方法如深度优先搜索(DFS)或广度优先搜索(BFS)能处理静态图,但当关系动态变化时(如频繁添加或删除好友),这些方法效率低下。每次查询都需要 O(n) 时间重建连通性,无法应对大规模数据。并查集(Disjoint Set Union)应运而生,它支持近常数时间的合并(union)与查询(find)操作,时间复杂度为 O(α(n)),其中 α(n) 是反阿克曼函数,增长极其缓慢;空间复杂度仅为 O(n)。本文将深入剖析并查集的核心原理,手把手实现两种关键优化(路径压缩和按秩合并),并通过实战代码解决算法问题。\par
\chapter{并查集核心概念剖析}
并查集的逻辑结构基于森林表示法:每个集合用一棵树表示,树根作为代表元(代表该集合)。初始时,每个元素自成集合;合并操作将两棵树连接,查询操作通过查找根节点判断元素所属集合。例如,元素 1、2、3 初始为独立集合,合并 1 和 2 后,它们共享同一个根。存储结构使用 \texttt{parent[]} 数组:\texttt{parent[i]} 存储元素 i 的父节点索引。初始化时,每个元素是自身的根,即 \texttt{parent[i] = i}。核心操作包括 \texttt{find(x)}(查找 x 的根)和 \texttt{union(x, y)}(合并 x 和 y 所在集合),这些操作确保了高效的动态处理能力。\par
\chapter{暴力实现与性能痛点}
基础版并查集未引入优化,代码简单但性能存在瓶颈。以下是 Python 实现:\par
\begin{lstlisting}[language=python]
class NaiveDSU:
    def __init__(self, n):
        self.parent = list(range(n))
    
    def find(self, x):
        while self.parent[x] != x:  # 暴力爬树:沿父节点链向上遍历
            x = self.parent[x]
        return x
    
    def union(self, x, y):
        rootX = self.find(x)
        rootY = self.find(y)
        if rootX != rootY:
            self.parent[rootY] = rootX  # 任意合并:可能导致树高度暴涨
\end{lstlisting}
在 \texttt{find} 方法中,通过 \texttt{while} 循环向上遍历父节点链,直到找到根节点。\texttt{union} 方法先调用 \texttt{find} 定位根节点,再将一个根指向另一个。问题在于:合并时若任意将小树挂到大树下,树可能退化成链表。例如,连续合并形成链式结构后,\texttt{find} 操作需遍历所有节点,时间复杂度恶化至 O(n),无法处理大规模操作(如 10\^{}6 次查询)。\par
\chapter{优化策略一:路径压缩(Path Compression)}
路径压缩的核心思想是在查询过程中扁平化访问路径,减少后续查询深度。具体分为两步变种:隔代压缩(在遍历时跳过父节点)和彻底压缩(递归压扁整个路径)。彻底压缩版效率更高,代码实现如下:\par
\begin{lstlisting}[language=python]
def find(self, x):
    if self.parent[x] != x:
        self.parent[x] = self.find(self.parent[x])  # 递归调用:将当前节点父指针直接指向根
    return self.parent[x]
\end{lstlisting}
在 \texttt{find} 方法中,递归调用 \texttt{self.find(self.parent[x])} 不仅返回根节点,还将 \texttt{x} 的父指针直接更新为根。例如,若路径为 x → p → root,递归后 x 和 p 都指向 root。这使树高度大幅降低,单次查询均摊时间复杂度优化至 O(α(n)),显著提升吞吐量。实际测试中,10\^{}6 次查询耗时从秒级降至毫秒级。\par
\chapter{优化策略二:按秩合并(Union by Rank)}
按秩合并通过控制树高度增长避免退化。秩(Rank)定义为树高度的上界(非精确高度),合并时总是将小树挂到大树下。代码增强如下:\par
\begin{lstlisting}[language=python]
class OptimizedDSU:
    def __init__(self, n):
        self.parent = list(range(n))
        self.rank = [0] * n  # 秩数组:初始高度为 0
    
    def union(self, x, y):
        rootX, rootY = self.find(x), self.find(y)
        if rootX == rootY: return
        
        if self.rank[rootX] < self.rank[rootY]:
            self.parent[rootX] = rootY  # 小树根指向大树根
        elif self.rank[rootX] > self.rank[rootY]:
            self.parent[rootY] = rootX
        else:  # 高度相同时
            self.parent[rootY] = rootX
            self.rank[rootX] += 1  # 更新秩:高度增加
\end{lstlisting}
在 \texttt{union} 方法中,比较根节点秩大小:若 \texttt{rank[rootX] < rank[rootY]},则将 rootX 挂到 rootY 下;高度相同时,任意合并并将新根的秩加 1。这确保树高度增长受控(最坏情况 O(log n)),避免链式结构。例如,合并两个高度为 2 的树时,新树高度为 3,而非暴力实现的随意增长。\par
\chapter{复杂度分析:反阿克曼函数之谜}
优化后(路径压缩 + 按秩合并),并查集操作的时间复杂度为 O(α(n))。α(n) 是反阿克曼函数,定义为阿克曼函数 A(n, n) 的反函数,增长极缓慢:在宇宙原子数(约 10\^{}80)范围内,α(n) < 5。数学上,阿克曼函数递归定义为:\par
$$  A(m, n) = \begin{cases} n+1 & \text{if } m = 0 \\ A(m-1, 1) & \text{if } m > 0 \text{ and } n = 0 \\ A(m-1, A(m, n-1)) & \text{otherwise} \end{cases}  $$\par
α(n) 是满足 A(k, k) ≥ n 的最小 k 值,其缓慢增长特性使并查集在工程中视为近常数时间。性能对比实验显示:10\^{}6 次操作下,未优化版耗时 >1000ms,优化版仅需 <50ms,差异显著。\par
\chapter{实战应用场景}
并查集在算法竞赛和工程中广泛应用。经典算法题如 LeetCode 547 朋友圈问题:给定 n×n 矩阵表示好友关系,求朋友圈数量。解法中初始化并查集,遍历矩阵,若 M[i][j] = 1 则调用 \texttt{union(i, j)},最后统计根节点数量。另一个场景是检测无向图环:遍历每条边,若 \texttt{find(u) == find(v)} 则存在环;否则调用 \texttt{union(u, v)}。这作为 Kruskal 最小生成树算法的前置步骤:排序边权重后,用并查集合并安全边。工程中,游戏地图动态计算连通区域(如玩家移动后更新区块连接),或编译器分析变量等价类(如类型推导),都依赖并查集的高效动态处理。\par
\chapter{完整代码实现(Python 版)}
以下是结合路径压缩和按秩合并的优化版并查集:\par
\begin{lstlisting}[language=python]
class DSU:
    def __init__(self, n):
        self.parent = list(range(n))  # 父指针数组:初始化每个元素自成一集合
        self.rank = [0] * n  # 秩数组:初始高度为 0
    
    def find(self, x):
        if self.parent[x] != x:
            self.parent[x] = self.find(self.parent[x])  # 路径压缩:递归压扁路径
        return self.parent[x]  # 返回根节点
    
    def union(self, x, y):
        rootX = self.find(x)  # 查找 x 的根
        rootY = self.find(y)  # 查找 y 的根
        if rootX == rootY:
            return False  # 已连通,无需合并
        
        if self.rank[rootX] < self.rank[rootY]:
            self.parent[rootX] = rootY  # 小树挂到大树下
        elif self.rank[rootX] > self.rank[rootY]:
            self.parent[rootY] = rootX
        else:
            self.parent[rootY] = rootX
            self.rank[rootX] += 1  # 高度相同时,新树高度 +1
        return True  # 合并成功
\end{lstlisting}
在 \texttt{find} 方法中,递归实现路径压缩,直接将路径节点指向根。\texttt{union} 方法使用秩比较:优先挂接小树,高度相同时更新秩。返回值 \texttt{True} 表示成功合并,便于外部逻辑跟踪。该实现时间复杂度 O(α(n)),空间 O(n),可直接用于解决算法问题。\par
\chapter{常见问题答疑(Q\&{}A)}
路径压缩和按秩合并可同时使用,因为两者正交:路径压缩优化查询路径,按秩合并优化合并策略;同时应用不会冲突,反而协同降低整体复杂度。秩是否可用节点数量替代?可以,称为重量合并(Union by Size),将小集合挂到大集合下,同样控制树高度;但高度合并(按秩)更精确避免高度暴涨。并查集本身不支持集合分裂;若需分裂操作,需扩展设计如维护反向指针,或改用其他数据结构如 Link-Cut Tree。\par
本文深入探讨了并查集的核心原理:森林表示法、\texttt{find}/\texttt{union} 操作、双优化策略(路径压缩和按秩合并),以及近常数时间复杂度 O(α(n))。实战中,它高效解决动态连通性问题,如社交网络或图算法。扩展学习建议包括带权并查集(处理关系传递问题,如「食物链」问题中距离权重)、动态并查集(支持删除操作,通过懒标记重建)、或并行并查集算法(分布式系统优化)。掌握这些,读者可进一步挑战复杂场景。\par
