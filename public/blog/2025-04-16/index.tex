\title{"6502 编程探秘"}
\author{"黄京"}
\date{"Apr 16, 2025"}
\maketitle
\chapter{导言}
在 Apple II 与 NES 等经典设备的黄金年代,程序员们用 1.79MHz 的 6502 处理器创造了无数奇迹。当现代开发者习惯于 GB 级内存时,这些先驱者却在 64KB 的「画布」上绘制出了精妙绝伦的代码画卷。本文将揭示如何通过精准的内存操控,让每个字节都发挥出最大效能。\par
\chapter{6502 内存架构速览}
6502 的寻址空间如同精密钟表,每个区域都有独特的设计哲学。零页(\${}0000-\${}00FF)的访问周期比普通内存少 1 个时钟周期,这看似微小的差异在循环中会产生惊人的累积效应。例如 \texttt{LDA \${}00} 仅需 3 个周期,而 \texttt{LDA \${}0100} 则需要 4 个周期。\par
栈空间的 256 字节限制催生了独特的编程范式。当处理器执行 \texttt{JSR} 指令时,返回地址被压入栈顶,但若在中断服务程序中过度使用栈空间,可能引发指针回绕灾难——这是许多早期游戏出现随机崩溃的元凶之一。\par
\chapter{零页攻防战}
零页是 6502 编程的兵家必争之地。优秀开发者会为高频变量保留零页地址:\par
\begin{lstlisting}[language=asm]
player_x    = $10  ; 零页地址分配
bullet_cnt  = $20
\end{lstlisting}
通过零页偏移可模拟额外寄存器。考虑以下伪寄存器扩展技巧:\par
\begin{lstlisting}[language=asm]
MACRO LOAD_ZP_INDEX idx
    LDA $F0, X  ; 当 F0 是零页基址时
ENDMACRO
\end{lstlisting}
动态重映射技术更将零页效用发挥到极致。在 NES 的《超级马里奥兄弟》中,通过 \texttt{MMC3} 芯片在飞行关卡时切换内存 bank,实现了零页空间的动态扩展。\par
\chapter{内存拓扑设计}
数据段规划需要遵循热力学第二定律——高频访问数据应靠近处理器。将精灵坐标放在 \${}0200-\${}02FF 区域,而背景音乐数据置于 \${}C000 区域,这种冷热分离策略可减少跨页访问。页面边界惩罚的数学表达式为:\par
$$ 周期惩罚 = \begin{cases} 0 & \text{当 } \text{addr}_{\text{新}}\ \&\ {\tt 0xFF00} = \text{addr}_{\text{旧}}\ \&\ {\tt 0xFF00} \\ 1 & \text{其他情况} \end{cases} $$\par
代码段优化则充满几何美感。将关键循环体置于内存中部的 \${}8000 地址,可使相对跳转指令 \texttt{BCC} 的覆盖范围最大化。一个经典的页面对齐案例:\par
\begin{lstlisting}[language=asm]
    ORG $8100      ; 确保子程序起始于页面边界
draw_sprite:
    ; 高频调用代码
\end{lstlisting}
\chapter{动态内存管理}
在 64KB 世界中,静态分配是首选策略。《魂斗罗》的关卡加载器在切换场景时执行批量释放:\par
\begin{lstlisting}[language=c]
// 伪代码示意
void load_stage(uint8_t stage) {
    release_all_bullets();
    dealloc(prev_stage_data);
    alloc(stage_data[stage]);
}
\end{lstlisting}
固定大小内存池是粒子系统的救星。以下 256 字节管理器的核心逻辑:\par
\begin{lstlisting}[language=asm]
mem_pool_init:
    LDA #<pool_start
    STA free_ptr
    LDA #>pool_start
    STA free_ptr+1  ; 初始化空闲指针

alloc_block:
    LDY #0
    LDA (free_ptr), Y  ; 读取下一空闲块地址
    STA temp_ptr
    INY
    LDA (free_ptr), Y
    STA temp_ptr+1
    ; 更新空闲指针 ...
\end{lstlisting}
\chapter{硬件协同优化}
DMA 时序是艺术与科学的结晶。在 NES 的垂直消隐期执行 \texttt{PPUADDR} 设置,可实现无闪烁的画面更新。音频双缓冲的实现关键:\par
\begin{lstlisting}[language=asm]
    LDA #<buffer1
    STA APU_ADDR
    LDA #>buffer1
    STA APU_ADDR    ; 填充后台缓冲
    ; 等待 VSYNC
    LDA active_buffer
    EOR #1
    STA active_buffer  ; 切换缓冲
\end{lstlisting}
内存镜像区域的写重定向技术,使得 Commodore 64 能在 \${}D000-\${}DFFF 区域通过 \texttt{\${}0001} 寄存器的第 0-2 位选择 I/O 设备,这种设计大幅减少了地址解码电路的复杂度。\par
\chapter{调试与逆向}
自制内存监视器是每个 6502 程序员的成人礼。以下断点处理代码可在触发时保存状态:\par
\begin{lstlisting}[language=asm]
break_handler:
    PHA
    TXA
    PHA
    TYA
    PHA          ; 保存寄存器
    LDA #<dump_area
    STA $FB
    LDA #>dump_area
    STA $FC      ; 设置存储地址
    LDY #0
dump_loop:
    LDA ($FD), Y ; $FD 存储监视地址
    STA ($FB), Y
    INY
    CPY #16
    BNE dump_loop
\end{lstlisting}
《超级马里奥兄弟》的对象池复用机制,将敌人结构体大小压缩到 16 字节,通过状态字段的位掩码实现多态行为,这种设计使得同屏 5 个敌人仅消耗 80 字节内存。\par
6502 的内存管理艺术在今天仍闪耀着智慧光芒。从物联网设备到航天器控制系统,这些诞生于 8 位时代的优化思想仍在继续传承。当你下次面对现代系统的海量内存时,不妨设想:若将其视为 64KB 的珍宝,是否能用更优雅的方式解决问题?\par
