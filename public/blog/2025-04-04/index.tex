\title{"JavaScript 中的事件循环机制"}
\author{"叶家炜"}
\date{"Apr 04, 2025"}
\maketitle
JavaScript 的单线程特性决定了它在处理异步任务时必须依赖事件循环机制。这一机制通过协调调用栈、内存堆和任务队列,实现了非阻塞的异步编程模型。例如,当发起一个网络请求时,浏览器不会等待响应返回,而是继续执行后续代码,待数据就绪后再通过回调函数处理结果。这种设计避免了主线程的阻塞,但也带来了执行顺序的复杂性。本文将深入剖析事件循环的核心原理,并探讨其在浏览器与 Node.js 中的差异及实践中的优化技巧。\par
\chapter{事件循环的核心原理}
\section{运行时环境的三要素}
JavaScript 的运行时环境由三部分组成:\textbf{调用栈(Call Stack)}、\textbf{内存堆(Heap)\textbf{和}任务队列(Task Queue)}。调用栈用于追踪函数的执行顺序,每个函数调用会形成一个栈帧;内存堆负责管理对象的动态内存分配;任务队列则存储待处理的异步任务回调。\par
事件循环的核心逻辑可简化为以下伪代码:\par
\begin{lstlisting}[language=javascript]
while (true) {
  if (调用栈为空) {
    const 任务 = 任务队列 . 取出下一个任务();
    执行(任务);
  }
}
\end{lstlisting}
\section{同步优先与异步分层}
同步代码始终优先执行,例如:\par
\begin{lstlisting}[language=javascript]
console.log('A');
setTimeout(() => console.log('B'), 0);
console.log('C');
// 输出顺序:A → C → B
\end{lstlisting}
\verb!setTimeout! 的回调被推入任务队列,等待调用栈清空后执行。异步任务进一步分为\textbf{宏任务}(如 \verb!setTimeout!)和\textbf{微任务}(如 \verb!Promise!),微任务在每轮事件循环的末尾优先执行。\par
\chapter{宏任务与微任务的执行规则}
\section{分类与优先级}
\textbf{宏任务}包括:\par
\begin{enumerate}
\item \verb!setTimeout!/\verb!setInterval!
\item I/O 操作
\item UI 渲染(浏览器)
\item \verb!setImmediate!(Node.js)
\end{enumerate}
\textbf{微任务}包括:\par
\begin{enumerate}
\item \verb!Promise.then!/\verb!async await!
\item \verb!MutationObserver!
\item \verb!process.nextTick!(Node.js,优先级高于普通微任务)
\end{enumerate}
\section{黄金执行顺序}
每轮事件循环处理一个宏任务后,会清空所有微任务队列。例如:\par
\begin{lstlisting}[language=javascript]
setTimeout(() => console.log(' 宏任务 1'), 0);
Promise.resolve().then(() => console.log(' 微任务 1'));
setTimeout(() => {
  console.log(' 宏任务 2');
  Promise.resolve().then(() => console.log(' 微任务 2'));
}, 0);
// 输出顺序:微任务 1 → 宏任务 1 → 微任务 2 → 宏任务 2
\end{lstlisting}
第一轮循环执行主线程代码(视为宏任务),触发微任务 \verb! 微任务 1!;随后处理 \verb! 宏任务 1!;下一轮处理 \verb! 宏任务 2! 时,其内部的 \verb! 微任务 2! 会立即执行。\par
\chapter{浏览器与 Node.js 的差异}
\section{浏览器的事件循环模型}
浏览器的事件循环与渲染管线紧密耦合。在一次循环中,可能包含以下步骤:\par
\begin{itemize}
\item 执行一个宏任务
\item 清空微任务队列
\item 执行 UI 渲染(如果需要)
\item 执行 \verb!requestAnimationFrame! 回调
\end{itemize}
这使得频繁的微任务可能延迟渲染,例如:\par
\begin{lstlisting}[language=javascript]
function 阻塞渲染() {
  Promise.resolve().then(阻塞渲染);
}
阻塞渲染();
// UI 更新会被无限延迟
\end{lstlisting}
\section{Node.js 的六阶段模型}
Node.js 基于 libuv 库实现事件循环,分为六个阶段:\par
\begin{itemize}
\item \textbf{Timers}:执行 \verb!setTimeout!/\verb!setInterval! 回调
\item \textbf{Pending Callbacks}:处理系统错误等挂起回调
\item \textbf{Idle/Prepare}:内部使用
\item \textbf{Poll}:检索新的 I/O 事件
\item \textbf{Check}:执行 \verb!setImmediate! 回调
\item \textbf{Close}:处理关闭事件(如 \verb!socket.on('close')!)
\end{itemize}
以下代码演示了 Node.js 中 \verb!setImmediate! 与 \verb!setTimeout! 的优先级:\par
\begin{lstlisting}[language=javascript]
setTimeout(() => console.log('setTimeout'), 0);
setImmediate(() => console.log('setImmediate'));
// 输出顺序可能不确定,取决于事件循环启动时间
\end{lstlisting}
\chapter{异步编程的最佳实践}
\section{从回调地狱到 async/await}
传统回调模式容易引发嵌套问题:\par
\begin{lstlisting}[language=javascript]
fs.readFile('A.txt', (err, dataA) => {
  fs.readFile('B.txt', (err, dataB) => {
    // 回调地狱
  });
});
\end{lstlisting}
使用 \verb!Promise! 和 \verb!async/await! 可扁平化代码:\par
\begin{lstlisting}[language=javascript]
async function 读取文件() {
  const dataA = await fs.promises.readFile('A.txt');
  const dataB = await fs.promises.readFile('B.txt');
  return [dataA, dataB];
}
\end{lstlisting}
\section{性能优化策略}
\begin{itemize}
\item \textbf{拆分长任务}:将耗时操作分解为多个微任务\begin{lstlisting}[language=javascript]
function 分片处理() {
  let i = 0;
  function 下一帧() {
    while (i < 1000 && 未超时) {
      // 处理数据
      i++;
    }
    if (i < 1000) {
      setTimeout(下一帧 , 0);
    }
  }
  下一帧();
}
\end{lstlisting}

\item \textbf{使用 Web Workers}:将 CPU 密集型任务转移到后台线程\begin{lstlisting}[language=javascript]
const worker = new Worker('task.js');
worker.postMessage(data);
worker.onmessage = (e) => console.log(e.data);
\end{lstlisting}

\end{itemize}
\chapter{案例解析}
\section{页面卡顿优化}
假设一个页面需要渲染 10,000 条数据,直接操作 DOM 会导致主线程阻塞:\par
\begin{lstlisting}[language=javascript]
// 错误示例
数据列表 .forEach(条目 => {
  const div = document.createElement('div');
  div.textContent = 条目 ;
  document.body.appendChild(div);
});
\end{lstlisting}
优化方案:使用 \verb!requestIdleCallback! 分批次处理\par
\begin{lstlisting}[language=javascript]
function 分片渲染(数据 , 索引 = 0) {
  requestIdleCallback((空闲时间) => {
    while (索引 < 数据 .length && 空闲时间 . 剩余时间() > 0) {
      创建元素(数据[索引]);
      索引 ++;
    }
    if (索引 < 数据 .length) {
      分片渲染(数据 , 索引);
    }
  });
}
\end{lstlisting}
事件循环机制是 JavaScript 异步编程的基石。理解宏任务与微任务的执行顺序、掌握浏览器与 Node.js 的差异,能够帮助开发者编写高效可靠的代码。随着 WebAssembly 和 Deno 等新技术的发展,异步模型仍在持续演进,但核心原理始终是构建复杂应用的指南针。\par
