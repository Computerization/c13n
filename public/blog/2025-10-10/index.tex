\title{深入理解并实现基本的深度优先搜索(DFS)算法}
\author{杨岢瑞}
\date{Oct 10, 2025}
\maketitle
想象你身处一个巨大的迷宫,没有地图,你该如何系统地探索每一条路径,确保不重不漏?一种最直观的策略便是“一条路走到黑”。遇到岔路时,选择一条路走到底,直到死胡同,然后返回上一个岔路口选择另一条路。这种“不撞南墙不回头”的策略,正是我们今天要深入探讨的深度优先搜索(Depth-First Search,DFS)算法的精髓。本文将带你从本质理解 DFS 的思想,掌握其递归与非递归两种实现方式,并通过经典问题学会如何应用它,从而在算法世界中游刃有余。\par
\chapter{DFS 核心思想解析}
深度优先搜索的核心思想可以用一个形象化的比喻来概括:“一条路走到黑”加上“后悔机制”(即回溯)。具体来说,DFS 在探索图或树结构时,会优先沿着一条路径深入到底,直到无法继续前进,然后回溯到上一个节点,尝试其他未探索的分支。这种策略依赖于栈(Stack)数据结构,无论是隐式的系统调用栈(在递归中)还是显式维护的栈(在迭代中),都利用了栈的“后进先出”特性来实现回溯过程。\par
DFS 的核心操作可以分解为几个关键步骤。首先,访问当前节点,例如打印节点信息或判断是否达到目标。其次,标记当前节点为已访问,这是防止程序在存在环的图中陷入无限循环的关键步骤。然后,对于当前节点的每一个未被访问的邻居节点,重复上述访问和标记过程,实现深度探索。最后,当所有邻居都被探索完毕时,回溯到上一个节点,继续探索其他分支。整个过程就像在迷宫中系统地尝试每一条路径,确保不会遗漏任何可能性。\par
\chapter{DFS 的两种实现方式}
\section{递归实现}
递归实现是 DFS 最直观和常用的方式,它利用函数调用栈来隐式地管理回溯过程。以下是一个基于图的邻接表表示的递归 DFS 实现框架,使用 Python 风格的伪代码。\par
\begin{lstlisting}[language=python]
def dfs_recursive(node, visited):
    # 标记当前节点为已访问
    visited[node] = True
    # 处理当前节点,例如打印节点信息
    print(f"Visiting node {node}")
    # 遍历当前节点的所有邻居
    for neighbor in graph[node]:
        # 如果邻居未被访问,递归调用 DFS
        if not visited[neighbor]:
            dfs_recursive(neighbor, visited)
\end{lstlisting}
在这段代码中,\texttt{visited} 数组用于记录每个节点的访问状态,防止重复访问。递归调用 \texttt{dfs\_{}recursive} 实现了深度优先的探索:每次函数调用都会处理当前节点,然后对其未访问的邻居递归调用自身,从而沿着一条路径深入到底。递归的终止条件隐含在循环中:当所有邻居都已被访问时,函数会自然返回,实现回溯。这种实现方式代码简洁,但需要注意递归深度过大时可能导致栈溢出。\par
\section{迭代实现}
迭代实现通过显式使用栈来模拟递归过程,避免了递归可能带来的栈溢出问题,并提供了更好的控制力。以下是迭代 DFS 的实现框架。\par
\begin{lstlisting}[language=python]
def dfs_iterative(start_node):
    # 初始化栈和已访问集合
    stack = []
    visited = set()
    # 将起始节点压入栈中
    stack.append(start_node)
    while stack:  # 当栈不为空时循环
        # 弹出栈顶元素
        node = stack.pop()
        if node not in visited:
            # 标记节点为已访问
            visited.add(node)
            # 处理当前节点
            print(f"Visiting node {node}")
            # 将未访问的邻居逆序压入栈中
            for neighbor in reversed(graph[node]):
                if neighbor not in visited:
                    stack.append(neighbor)
\end{lstlisting}
在迭代实现中,栈用于存储待访问的节点,\texttt{visited} 集合记录已访问节点。关键点在于,我们在弹出节点后才检查其访问状态,这是因为同一节点可能被多次压入栈中(例如通过不同路径)。逆序压入邻居节点是为了保持与递归实现一致的遍历顺序,否则由于栈的后进先出特性,遍历顺序可能会颠倒。迭代实现虽然代码稍复杂,但能有效控制内存使用,适用于深度极大的场景。\par
递归实现和迭代实现在可读性、性能和适用场景上各有优劣。递归实现代码简洁,更符合 DFS 的思维模型,但在深度过大时可能引发栈溢出。迭代实现通过显式栈避免了这一问题,且便于控制执行流程,但代码可读性稍低。在实际应用中,如果图深度可控,递归实现是首选;对于深度极大或需要精细控制栈的场景,迭代实现更为合适。无论哪种方式,核心都是利用栈的后进先出特性来实现深度优先和回溯。\par
\chapter{DFS 的实战应用}
\section{应用一:二叉树的前序遍历}
二叉树的前序遍历是 DFS 在树结构上的直接体现,访问顺序为“根节点 -> 左子树 -> 右子树”。以下分别用递归和迭代实现前序遍历。\par
递归实现代码如下:\par
\begin{lstlisting}[language=python]
def preorder_recursive(root):
    if root is None:
        return
    # 访问根节点
    print(root.val)
    # 递归遍历左子树
    preorder_recursive(root.left)
    # 递归遍历右子树
    preorder_recursive(root.right)
\end{lstlisting}
在这段代码中,我们首先处理当前节点(根节点),然后递归处理左子树和右子树,这正符合 DFS 的深度优先思想。递归调用栈确保了回溯的正确性。\par
迭代实现代码如下:\par
\begin{lstlisting}[language=python]
def preorder_iterative(root):
    if root is None:
        return
    stack = [root]
    while stack:
        node = stack.pop()
        # 访问当前节点
        print(node.val)
        # 先将右子节点压栈,再将左子节点压栈,以确保左子树先被处理
        if node.right:
            stack.append(node.right)
        if node.left:
            stack.append(node.left)
\end{lstlisting}
迭代实现使用栈来模拟递归过程。由于栈的后进先出特性,我们先将右子节点压栈,再将左子节点压栈,这样左子节点会先被弹出和处理,实现了前序遍历的顺序。这种方法展示了 DFS 如何通过栈管理遍历路径。\par
\section{应用二:查找路径}
在图中查找从节点 A 到节点 B 的路径是 DFS 的经典应用。我们可以通过 DFS 探索所有可能路径,并记录访问顺序。以下是一个查找路径的递归实现示例。\par
\begin{lstlisting}[language=python]
def find_path_dfs(start, target, visited, path):
    # 标记当前节点为已访问
    visited[start] = True
    # 将当前节点加入路径
    path.append(start)
    if start == target:
        # 找到目标,返回路径
        return path.copy()
    # 遍历邻居节点
    for neighbor in graph[start]:
        if not visited[neighbor]:
            # 递归搜索
            result = find_path_dfs(neighbor, target, visited, path)
            if result:
                return result
    # 回溯:从路径中移除当前节点
    path.pop()
    return None
\end{lstlisting}
在这个实现中,我们使用 \texttt{visited} 数组避免重复访问,\texttt{path} 列表记录当前路径。当找到目标节点时,返回路径副本;否则,在回溯时从路径中移除当前节点。这体现了 DFS 的回溯机制,但注意 DFS 找到的路径不一定是最短路径,因为它优先深度探索。\par
\chapter{DFS 的复杂度分析与常见陷阱}
深度优先搜索的时间复杂度通常为 $O(V + E)$,其中 $V$ 是顶点数,$E$ 是边数。这是因为每个节点和边最多被访问一次。空间复杂度主要取决于 \texttt{visited} 数据结构和栈的使用,最坏情况下为 $O(V)$,例如当图呈链状结构时,栈可能需要存储所有节点。\par
在实际实现中,常见的陷阱包括忘记标记节点为已访问,这会导致在存在环的图中陷入无限循环;在迭代实现中,在错误的位置标记已访问可能导致节点被重复处理;此外,混淆 DFS 与广度优先搜索(BFS)的适用场景也是一个常见错误,例如 DFS 不适合求解非加权图的最短路径问题。为了避免这些陷阱,务必确保在访问节点时立即标记,并根据问题特性选择合适的算法。\par
深度优先搜索的核心在于其“深度优先”和“回溯”思想,以及栈数据结构的巧妙运用。DFS 的优点包括实现简单、空间效率较高,并且是许多高级算法(如回溯和 Tarjan 算法)的基础;缺点则是可能无法找到最优解(如最短路径),且递归实现有深度限制。\par
进一步思考,DFS 与回溯算法密切相关,回溯本质上是 DFS 加上剪枝优化。与 BFS 相比,DFS 更适用于探索所有可能解或路径存在的场景,而 BFS 则擅长寻找最短路径。鼓励读者动手实现代码,并尝试应用 DFS 解决更复杂的问题,如数独或 N 皇后问题,从而深化对算法的理解。通过不断实践,你将能灵活运用 DFS 应对各种挑战。\par
