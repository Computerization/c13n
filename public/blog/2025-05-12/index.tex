\title{"深入理解词嵌入技术原理与应用实践"}
\author{"黄京"}
\date{"May 12, 2025"}
\maketitle
自然语言处理(NLP)的核心挑战在于语言的\textbf{非结构化特性}与\textbf{语义复杂性}。传统方法如独热编码(One-Hot Encoding)和词袋模型(Bag of Words)仅能捕捉表面统计信息,无法处理同义词、一词多义等语义关联。例如,独热编码将每个词映射为高维稀疏向量,导致「猫」与「犬」的向量距离和「猫」与「汽车」的向量距离相同,显然违背语义直觉。\par
词嵌入(Embeddings)技术通过将词汇映射到\textbf{低维稠密向量空间},实现了从符号表示到分布式表示的范式跃迁。这一技术革命性地解决了语义相似性与上下文关联性的建模问题。例如,在向量空间中,「国王」-「男性」+「女性」≈「女王」的向量关系,直观展示了词嵌入对语义关系的几何表达。\par
\chapter{词嵌入技术原理}
\section{基础概念}
词嵌入的核心目标是将词汇从高维稀疏向量(如独热编码的维度等于词表大小)映射到低维稠密向量(通常为 50-300 维)。这种映射使得语义相似的词在向量空间中距离相近。例如,「快乐」与「愉快」的余弦相似度应显著高于「快乐」与「悲伤」。实现这一目标的关键在于\textbf{上下文关联性}:通过分析词汇在语料中的共现模式,模型能够学习到词汇的分布式表示。\par
\section{经典模型解析}
\textbf{Word2Vec}是词嵌入领域的里程碑模型,其包含两种架构:Skip-Gram 与 CBOW。Skip-Gram 通过中心词预测上下文词,适合处理低频词;而 CBOW 通过上下文词预测中心词,训练效率更高。两者的损失函数均基于极大似然估计:\par
$$ L = -\sum_{c \in \text{Context}} \log p(w_c | w_t) $$\par
其中 $w_t$ 为中心词,$w_c$ 为上下文词。为降低计算复杂度,Word2Vec 引入\textbf{负采样}(Negative Sampling)技术,将多分类问题转化为二分类问题。例如,对于正样本(中心词与真实上下文词对),模型输出概率应接近 1;对于随机采样的负样本,输出概率应接近 0。\par
\textbf{GloVe}(Global Vectors)则从全局词共现矩阵出发,通过优化目标函数直接学习词向量:\par
$$ J = \sum_{i,j=1}^V f(X_{ij}) (w_i^T \tilde{w}_j + b_i + \tilde{b}_j - \log X_{ij})^2 $$\par
其中 $X_{ij}$ 表示词 $i$ 与词 $j$ 的共现次数,$f(X_{ij})$ 为加权函数,用于抑制高频词的影响。\par
\section{上下文感知的嵌入技术}
传统词嵌入模型生成\textbf{静态向量},无法处理一词多义问题。例如,「苹果」在「吃苹果」与「苹果手机」中的语义差异无法通过单一向量表达。\textbf{ELMo}(Embeddings from Language Models)通过双向 LSTM 生成动态嵌入,结合不同网络层的表示,捕捉词汇的多层次语义。而\textbf{BERT}(Bidirectional Encoder Representations from Transformers)基于 Transformer 的注意力机制,通过掩码语言模型(Masked Language Model)预训练,生成上下文相关的词向量。例如:\par
\begin{lstlisting}[language=python]
from transformers import BertTokenizer, BertModel
tokenizer = BertTokenizer.from_pretrained('bert-base-uncased')
model = BertModel.from_pretrained('bert-base-uncased')
inputs = tokenizer("bank of the river", return_tensors="pt")
outputs = model(**inputs)
word_embeddings = outputs.last_hidden_state
\end{lstlisting}
此代码加载预训练 BERT 模型,对句子「bank of the river」进行编码。\verb!last_hidden_state! 输出包含每个 token 的上下文相关向量,其中「bank」的向量会根据「river」的上下文动态调整,从而区别于「bank account」中的「bank」。\par
\chapter{词嵌入的应用实践}
\section{基础 NLP 任务}
在文本分类任务中,可通过简单平均所有词向量得到句子表示,再输入全连接网络进行分类。例如,使用 Gensim 训练 Word2Vec 模型:\par
\begin{lstlisting}[language=python]
from gensim.models import Word2Vec
sentences = [["自然", "语言", "处理", "是", "人工智能", "的", "核心"], ["词嵌入", "技术", "推动", "了", "NLP", "发展"]]
model = Word2Vec(sentences, vector_size=100, window=5, min_count=1, workers=4)
sentence_vector = np.mean([model.wv[word] for word in ["词嵌入", "技术", "NLP"]], axis=0)
\end{lstlisting}
此处 \verb!vector_size! 定义词向量维度,\verb!window! 控制上下文窗口大小。通过 \verb!np.mean! 对词向量取平均,得到句子级表示。\par
\section{高级应用场景}
在语义搜索场景中,可通过余弦相似度匹配用户查询与文档向量。例如,将用户查询「智能语音助手」与文档库中的向量进行相似度排序,返回最相关结果。对于跨语言任务,\textbf{Facebook MUSE}项目通过对抗训练对齐不同语言的向量空间,使得「dog」的向量与「犬」的向量在映射后接近。\par
\section{实战代码示例}
使用 t-SNE 对词向量降维可视化:\par
\begin{lstlisting}[language=python]
from sklearn.manifold import TSNE
import matplotlib.pyplot as plt

words = ["国王", "女王", "男人", "女人", "巴黎", "法国"]
vectors = [model.wv[word] for word in words]
tsne = TSNE(n_components=2, random_state=0)
projections = tsne.fit_transform(vectors)

plt.figure(figsize=(10, 6))
for i, word in enumerate(words):
    plt.scatter(projections[i, 0], projections[i, 1])
    plt.annotate(word, xy=(projections[i, 0], projections[i, 1]))
plt.show()
\end{lstlisting}
此代码将高维词向量投影到二维平面,\verb!n_components=2! 指定输出维度为 2。可视化结果可清晰展示「国王-女王-男人-女人」的性别语义轴与「巴黎-法国」的地理关联。\par
\chapter{挑战与优化方向}
静态词嵌入的核心局限在于无法处理\textbf{一词多义}与\textbf{领域迁移}问题。例如,「细胞」在生物学与计算机领域分别指向「生物单元」与「电子元件」。动态嵌入模型如 BERT 虽能缓解此问题,但计算成本较高。优化策略包括\textbf{领域自适应}(Domain Adaptation):在目标领域数据上微调预训练模型,使其适应特定术语分布。例如,在医疗文本上微调 BERT:\par
\begin{lstlisting}[language=python]
from transformers import BertForMaskedLM, Trainer, TrainingArguments
model = BertForMaskedLM.from_pretrained('bert-base-uncased')
training_args = TrainingArguments(
    output_dir='./med_bert',
    overwrite_output_dir=True,
    num_train_epochs=3,
    per_device_train_batch_size=16,
)
trainer = Trainer(
    model=model,
    args=training_args,
    train_dataset=medical_dataset
)
trainer.train()
\end{lstlisting}
通过 3 轮训练,模型能够学习医疗领域特有的语义模式,提升在该领域的下游任务表现。\par
\chapter{未来展望}
多模态嵌入将文本、图像、语音的表示统一到同一空间,例如 OpenAI 的 CLIP 模型,可将「狗」的文本描述与狗的图像映射到相近向量。在可解释性方向,\textbf{Embedding Projector}等工具允许用户交互式探索高维向量空间,分析模型语义捕获能力。轻量化技术如模型蒸馏(Distillation)可将 BERT 压缩为 TinyBERT,在保持 90\%{} 性能的同时减少 70\%{} 参数量,推动词嵌入技术在移动端的落地。\par
