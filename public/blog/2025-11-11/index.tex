\title{深入理解并实现基本的基数排序(Radix Sort)算法}
\author{李睿远}
\date{Nov 11, 2025}
\maketitle
在计算机科学中,排序算法是基础且关键的主题。常见的比较排序算法如快速排序和归并排序,其时间复杂度通常为 O(n log n),这是基于比较操作的理论下限。然而,是否存在一种方法能够突破「比较」这一范式,实现更优的性能呢?答案是肯定的,基数排序作为一种非比较型整数排序算法,通过逐位处理数字,可以在特定条件下达到线性时间复杂度。本文将深入解析基数排序的核心原理、实现方式及其应用,帮助你全面掌握这一高效算法。\par
\chapter{基数排序初探}
基数排序的核心思想在于将待排序元素视为由多个「位」组成的序列,而不是直接比较整体大小。具体来说,算法从最低位或最高位开始,依次对每一位进行稳定的排序操作。这个过程可以类比为整理扑克牌:如果我们先按点数分类,再在同点数内按花色排序,就类似于最低位优先方法;反之,如果先按花色分,再按点数排序,则类似于最高位优先方法。这种分步处理的方式使得基数排序能够避免直接比较元素,从而在整数或字符串排序中展现出独特优势。\par
基数排序的关键特性之一是其对稳定性的依赖。稳定性指的是在排序过程中,相等元素的相对顺序保持不变。基数排序的每一轮排序都必须使用稳定的次级算法,通常选择计数排序,因为如果次级排序不稳定,整个算法的正确性将无法保证。此外,基数排序主要适用于整数或可以分解为「位」的数据类型,如字符串。对于浮点数或其他复杂类型,则需要额外处理,例如通过转换或偏移来适应算法要求。\par
\chapter{深入原理}
最低位优先(LSD)方法是基数排序中最常见的实现方式。它从数字的最低位(如个位)开始,依次向高位进行稳定排序。例如,给定数组 \texttt{[170, 45, 75, 90, 2, 802, 24, 66]},LSD 会先按个位排序,结果可能为 \texttt{[170, 90, 2, 802, 24, 45, 75, 66]};接着按十位排序,得到 \texttt{[2, 802, 24, 45, 66, 170, 75, 90]};最后按百位排序,完成整个排序过程。LSD 实现简单直观,但它是一种离线算法,需要预先知道最大数字的位数,以确定排序轮数。排序结束后,序列自然有序,无需额外合并步骤。\par
最高位优先(MSD)方法则从最高位开始排序,然后递归处理每个子桶。例如,对同一数组,MSD 会先按百位分桶,将数字分配到不同范围,再对每个桶内的数字递归排序低位。这种方法更接近分治策略,可能在某些情况下提前终止,例如当某个桶内只有一个元素时。MSD 在字符串字典序排序中尤为自然,因为它可以直接处理前缀。与 LSD 相比,MSD 在实现上可能更复杂,且性能受数据分布影响较大,但它能更早地排除无关比较。\par
\chapter{代码实现}
在实现基数排序时,我们通常选择计数排序作为稳定的次级排序算法。计数排序通过统计每个数字的出现次数,并利用前缀和确定元素位置,从而保证稳定性。以下以 LSD 方法为例,使用 Python 语言实现基数排序。代码将分步解释,确保每个细节清晰易懂。\par
首先,我们需要确定最大数字的位数,以决定排序轮数。这可以通过遍历数组并计算最大值来实现。例如,如果最大数字是 802,其位数为 3,则需进行三轮排序(个位、十位、百位)。\par
\begin{lstlisting}[language=python]
def radix_sort(arr):
    # 步骤 1: 寻找最大数,确定位数
    max_num = max(arr)
    exp = 1  # 从个位开始
    while max_num // exp > 0:
        # 使用计数排序对当前位进行排序
        n = len(arr)
        output = [0] * n
        count = [0] * 10  # 十进制数字范围 0-9
        
        # a. 计数:统计当前位上每个数字的出现次数
        for i in range(n):
            index = (arr[i] // exp) % 10
            count[index] += 1
        
        # b. 计算位置:将计数转换为前缀和,表示起始索引
        for i in range(1, 10):
            count[i] += count[i - 1]
        
        # c. 构建输出:从后向前遍历,保证稳定性
        i = n - 1
        while i >= 0:
            index = (arr[i] // exp) % 10
            output[count[index] - 1] = arr[i]
            count[index] -= 1
            i -= 1
        
        # d. 复制回原数组
        for i in range(n):
            arr[i] = output[i]
        
        exp *= 10  # 移动到下一位
\end{lstlisting}
在这段代码中,我们首先计算最大数字以确定循环次数。然后,在每一轮中,我们使用计数排序处理当前位。计数步骤统计每个数字(0-9)的出现频率;位置计算步骤将计数数组转换为前缀和,以确定每个数字在输出数组中的起始索引;构建输出步骤从原数组末尾开始遍历,确保相等元素的顺序不变;最后,将结果复制回原数组。关键点在于从后向前遍历,这维护了稳定性,因为计数排序中后出现的元素会被放置在输出数组的较后位置。每轮结束后,\texttt{exp} 乘以 10,以处理更高位。\par
实现基数排序时,常见的陷阱包括忽略稳定性或错误处理数字位。例如,如果构建输出时从前向后遍历,可能会破坏相对顺序。另外,确保 \texttt{exp} 正确递增,避免遗漏高位或重复处理。\par
\chapter{算法分析}
基数排序的时间复杂度为 O(d * (n + k)),其中 d 是最大数字的位数,n 是元素个数,k 是每位可能取值的范围(对于十进制,k=10)。当 d 为常数且 k 与 n 同阶时,时间复杂度可视为线性 O(n)。相比之下,比较排序算法如快速排序的平均时间复杂度为 O(n log n),基数排序在特定条件下更具优势。例如,如果数字范围有限且位数较少,基数排序能显著提升性能。\par
空间复杂度主要来自计数排序的辅助数组,包括计数数组和输出数组,因此为 O(n + k)。这表示基数排序不是原地算法,需要额外内存空间。尽管这可能成为内存受限环境中的缺点,但其稳定性和高效性在许多应用中值得权衡。\par
基数排序的优缺点总结如下:优点包括在整数排序中可能达到线性时间、稳定性高;缺点则在于适用范围有限、需要额外空间,且当数字位数差异大时,性能可能不如某些自适应比较排序。因此,在选择算法时,需考虑数据特性和环境约束。\par
基数排序在实际应用中常用于处理固定位数的整数序列,例如身份证号、电话号码排序,或字符串字典序排列。在计算机图形学中,它也可用于某些像素处理算法。这些场景充分利用了基数排序的稳定性和高效性。\par
总结来说,基数排序通过「按位排序」和「依赖稳定性」的核心思想,实现了非比较排序的突破。其线性时间复杂度的优势在合适条件下显著,但需注意数据类型的限制。鼓励读者在涉及整数或字符串排序的任务中,尝试应用这一算法。\par
思考与拓展部分留给读者进一步探索:例如,如何对包含负数的数组进行基数排序?可以通过分离正负数组或使用偏移量处理;MSD 实现需调整代码结构为递归形式;对于非十进制数字,只需修改进制基数即可适应。这些拓展问题有助于深化对算法灵活性的理解。\par
