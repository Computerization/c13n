\title{深入理解并实现基本的 JSON 解析器}
\author{黄京}
\date{Sep 21, 2025}
\maketitle
\chapter{副标题:抛开现成的库,亲手打造一个解析器,彻底掌握 JSON 的本质。}
JSON(JavaScript Object Notation)作为现代数据交换的事实标准,几乎无处不在。从 Web API 的响应到配置文件的存储,JSON 以其轻量级和易读性赢得了广泛的应用。我们日常开发中经常使用如 Python 的 \texttt{json.loads()} 或 JavaScript 的 \texttt{JSON.parse()} 来解析 JSON 数据,但这些现成库的背后机制却往往被忽略。超越简单的导入和使用,亲手实现一个 JSON 解析器,不仅能帮助我们深入理解编译原理的基础知识如语法、词法和状态机,还能在遇到非标准或自定义数据格式时提供自主解决问题的能力。此外,这个过程极大地锻炼了编程技能、对细节的把握和调试能力。本文的目标是使用 Python 语言从零实现一个功能完备的 JSON 解析器,并通过官方的 JSONTestSuite 进行测试,以确保其正确性和健壮性。\par
\chapter{背景知识:JSON 格式规范与解析器概述}
JSON 的语法规范基于 RFC 7159,其基本结构包括对象(用花括号 \texttt{\{{}\}{}} 表示)和数组(用方括号 \texttt{[]} 表示),以及基本类型如字符串、数字、布尔值(\texttt{true} 或 \texttt{false})和 \texttt{null}。重要规则包括键必须用双引号包裹、禁止尾部逗号等。解析 JSON 字符串通常涉及两个核心步骤:词法分析和语法分析。词法分析负责将字符流分解为有意义的词元(Token),类似于将句子拆分成单词;语法分析则根据 Token 流按照语法规则构建内存中的数据结构,如 Python 的字典或列表。整个解析器的架构可以概括为:JSON 字符串输入到词法分析器(Lexer)生成 Token 流,再传递给语法分析器(Parser)输出 Python 对象。\par
\chapter{第一步:构建词法分析器(Lexer)}
词法分析器的核心任务是识别和生成 Token。Token 类型包括结构字符如 \texttt{\{{}}, \texttt{\}{}}, \texttt{[}, \texttt{]}, \texttt{:}, \texttt{,},以及值类型如 \texttt{STRING}, \texttt{NUMBER}, \texttt{TRUE}, \texttt{FALSE}, \texttt{NULL}。Lexer 的工作流程是循环遍历输入字符串,使用条件判断和状态机来识别这些 Token。实现过程中,需要处理空白字符(如空格、制表符、换行符)的跳过,以及解析字符串时的转义序列(例如 \texttt{\textbackslash{}\textbackslash{}}, \texttt{\textbackslash{}"}, \texttt{\textbackslash{}n}, \texttt{\textbackslash{}uXXXX})。数字解析涉及识别负号、整数部分、小数部分和指数部分,策略通常是持续读取相关字符后统一用 \texttt{float()} 转换。字面量如 \texttt{true}, \texttt{false}, \texttt{null} 则通过匹配关键字来识别。\par
以下是一个简单的 Lexer 类框架代码示例:\par
\begin{lstlisting}[language=python]
class Lexer:
    def __init__(self, input_string):
        self.input = input_string
        self.position = 0
        self.current_char = self.input[self.position] if self.input else None

    def advance(self):
        self.position += 1
        if self.position < len(self.input):
            self.current_char = self.input[self.position]
        else:
            self.current_char = None

    def skip_whitespace(self):
        while self.current_char is not None and self.current_char.isspace():
            self.advance()

    def next_token(self):
        # 实现 Token 识别逻辑
        pass
\end{lstlisting}
在这个代码中,\texttt{\_{}\_{}init\_{}\_{}} 方法初始化输入字符串和当前位置,\texttt{advance} 方法移动指针到下一个字符,\texttt{skip\_{}whitespace} 用于跳过空白字符。\texttt{next\_{}token} 方法是核心,它根据当前字符判断 Token 类型。例如,如果字符是 \texttt{\{{}},则返回一个表示左花括号的 Token;如果是双引号,则开始解析字符串。字符串解析需要处理转义序列,这是一个难点,因为必须正确识别和转换如 \texttt{\textbackslash{}n} 为换行符。数字解析则涉及收集所有数字相关字符(包括符号、小数点、指数),然后使用 \texttt{float()} 进行转换,但需要注意错误处理,例如无效数字格式。\par
\chapter{第二步:构建语法分析器(Parser)}
语法分析器采用递归下降解析法,这是一种直观的方法,适合 JSON 这种上下文无关文法。每个语法规则对应一个解析函数。入口函数是 \texttt{parse()},它根据第一个 Token 决定解析为对象或数组。\texttt{parse\_{}object()} 函数处理花括号内的键值对,循环读取直到遇到右花括号;\texttt{parse\_{}array()} 函数处理方括号内的值列表;\texttt{parse\_{}value()} 是核心分发函数,根据当前 Token 类型调用相应的解析函数,如 \texttt{parse\_{}string} 或 \texttt{parse\_{}number},或递归调用自身处理嵌套结构。\par
错误处理是关键部分,例如在预期冒号分隔键值对时却遇到其他 Token,应抛出清晰异常如“Expected ‘:’ after key”。解析器需要与 Lexer 协同工作,确保在解析完一个结构后,Lexer 的位置正确指向下一个 Token。以下是一个 Parser 类的框架代码:\par
\begin{lstlisting}[language=python]
class Parser:
    def __init__(self, lexer):
        self.lexer = lexer
        self.current_token = self.lexer.next_token()

    def eat(self, token_type):
        if self.current_token.type == token_type:
            self.current_token = self.lexer.next_token()
        else:
            raise Exception(f"Expected {token_type}, got {self.current_token.type}")

    def parse(self):
        if self.current_token.type == 'LBRACE':
            return self.parse_object()
        elif self.current_token.type == 'LBRACKET':
            return self.parse_array()
        else:
            return self.parse_value()

    def parse_object(self):
        # 解析对象逻辑
        pass
\end{lstlisting}
在这个代码中,\texttt{\_{}\_{}init\_{}\_{}} 方法接收 Lexer 实例并获取第一个 Token。\texttt{eat} 方法用于消耗预期类型的 Token,如果类型不匹配则抛出错误。\texttt{parse} 方法根据当前 Token 类型决定解析方向。\texttt{parse\_{}object} 函数会循环读取键值对,每次读取一个字符串键、冒号、值,并处理逗号分隔。递归下降法的优势在于代码结构清晰,易于理解和调试,但需要 careful handling of recursive calls to avoid infinite loops。\par
\chapter{整合与测试}
将 Lexer 和 Parser 整合为一个函数 \texttt{my\_{}json\_{}loads(s)},它接收 JSON 字符串并返回 Python 对象。基础测试用例包括简单 JSON 如 \texttt{\{{}"key": "value"\}{}},应解析为字典;数组如 \texttt{[1, true, null]},应解析为列表。挑战性测试涉及嵌套结构,例如多层对象或数组,以验证递归处理的正确性。错误处理测试包括非法输入如缺少逗号或键名无双引号,解析器应提供有用的错误信息而非崩溃。最后,引入 JSONTestSuite 进行自动化测试,确保解析器符合官方标准,例如处理边缘情况如空字符串或超大数字。\par
通过实现 JSON 解析器,我们完整经历了从字符串到 Token 再到数据结构的解析过程,加深了对词法分析、语法分析和递归下降法的理解。未来优化方向包括性能提升(如使用生成器惰性产生 Token)、功能扩展(添加自定义参数如 \texttt{parse\_{}float})和增强鲁棒性(改进错误恢复机制)。鼓励读者以此为基础,探索更复杂格式如 XML 或 TOML 的解析,进一步提升编译原理技能。\par
\chapter{附录 \&{} 进一步阅读}
完整代码可参考 GitHub 仓库示例。参考资源包括 RFC 7159 标准文档、JSONTestSuite 项目以及经典书籍《编译原理》(俗称龙书)的相关章节。这些资料有助于深入理解解析器设计和实现细节。\par
