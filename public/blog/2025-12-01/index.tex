\title{Java 垃圾回收机制详解}
\author{王思成}
\date{Dec 01, 2025}
\maketitle
Java 垃圾回收机制是 Java 虚拟机(JVM)内存管理的最核心特性之一,它自动识别并回收不再使用的对象内存,避免了开发者手动管理内存的复杂性和错误风险。在现代 Java 应用中,尤其是在高并发、大规模分布式系统中,GC 的性能直接影响系统的吞吐量、延迟和稳定性。本文将从 JVM 内存结构入手,深入剖析垃圾判定机制、回收算法、分代策略、主流收集器实现,以及参数调优实战,帮助读者系统掌握 Java GC 的全貌。\par
为什么需要垃圾回收呢?在像 C/C++ 这样的语言中,开发者必须手动分配内存(如使用 malloc 或 new)和释放内存(如 free 或 delete),稍有不慎就会导致内存泄漏或悬垂指针问题。Java 通过 GC 自动处理这些事务,大大提高了开发效率和代码可靠性,但也引入了 GC 暂停(Stop-The-World,简称 STW)等性能开销。GC 的核心目标不仅是回收无用对象,还需在内存回收效率与应用暂停时间之间取得平衡,实现高吞吐量和低延迟的双重优化。\par
本文结构清晰,先介绍 JVM 内存基础,再探讨垃圾判定与算法,然后深入分代回收和收集器实现,最后聚焦调优实战和日志分析。适合有 Java 基础和 JVM 概述知识的开发者阅读,无论你是初学者想理解 GC 原理,还是架构师需要优化生产环境,都能从中获益。\par
\chapter{2. JVM 内存结构基础}
JVM 运行时数据区是理解 GC 的基石,它分为线程私有和线程共享区域。线程私有区域包括程序计数器(PC Register),用于记录当前线程执行的字节码指令地址;虚拟机栈,存储局部变量表、操作数栈等;本地方法栈,则服务于 Native 方法调用。这些区域随线程生命周期而生灭,不涉及 GC。\par
线程共享区域则包括方法区(JDK 8 前为永久代,之后演变为 Metaspace,用于存储类元数据、常量池等)和堆(Heap),后者是 GC 的主要战场。堆内存按对象生命周期分代设计,新生代(Young Generation)存放短生命周期对象,老年代(Old Generation)存放长生命周期对象,而 Metaspace 虽不直接回收对象,但其溢出会间接触发 Full GC。\par
堆内存的核心是新生代,由 Eden 区和两个 Survivor 区(S0 和 S1)组成。新对象优先分配在 Eden,当 Eden 满时触发 Minor GC,存活对象复制到 Survivor,经历多次 GC 后年龄达到阈值(如 15)则晋升老年代。这种分代设计基于「大多数对象朝生夕死」的经验假设,大幅提升了 GC 效率。\par
\chapter{3. 垃圾判定机制}
垃圾对象是指应用不再可达的对象,即没有有效引用指向它。Java 通过引用类型和可达性分析来判定垃圾。从 JDK 8 开始,引用分为强引用、软引用、弱引用和虚引用。强引用是最常见的,如 Object obj = new Object(),只要强引用存在,GC 绝不回收。软引用在内存不足时才回收,常用于图片缓存;弱引用在 GC 时无条件回收,适合内存敏感的缓存场景;虚引用最弱,仅用于监控对象销毁,无法通过它访问对象,是 finalizer 的现代替代。\par
垃圾判定主要依赖可达性分析算法,从 GC Roots 出发遍历对象图。GC Roots 包括虚拟机栈中的局部变量、方法区中的静态变量、常量、JNI 句柄等。以一个简单示例说明:\par
\begin{lstlisting}[language=java]
public class GCRootsDemo {
    public static Object root = new Object();  // GC Root: 静态变量
    public void method() {
        Object local = new Object();  // GC Root: 局部变量
        root = local;  // local 通过 root 可达
    }
}
\end{lstlisting}
在这里,root 是静态变量,故为 GC Root;method 中的 local 是栈帧局部变量,也为 GC Root。通过 root 引用,local 对象可达,不会回收。如果移除 root = local,则 local 在方法返回后不可达,成为垃圾。引用计数算法虽简单(每个对象计数器加减),但无法处理循环引用,故 HotSpot JVM 弃用它,转用可达性分析。\par
\chapter{4. 垃圾回收算法详解}
GC 过程分为标记(Mark)和清除/整理(Sweep/Compact)阶段。标记阶段从 GC Roots 遍历,标记存活对象。标记-清除算法最简单,先标记再清除未标记对象,但会产生内存碎片,导致分配大对象时失败。标记-整理则在清除后移动存活对象,消除碎片,但需要额外整理时间,STW 更长。标记-复制适用于对象存活率低的场景,将内存分为两块,复制存活对象到另一块后清空原块,简单高效但空间利用率仅 50\%{}。\par
新生代常用标记-复制,老年代偏好标记-整理。并发时代引入三色标记算法:对象分为白(未标记)、灰(标记待扫描)和黑(已标记完成)。并发标记时,用户线程可能移动对象,导致「漏标」问题。为此,CMS 使用增量更新,G1 采用 Snapshot-At-The-Beginning(SATB)屏障,确保正确性。这些算法在 $\mathcal{O}(n)$ 时间复杂度内完成标记,其中 $n$ 为对象数。\par
\chapter{5. 分代回收策略与 GC 类型}
分代回收基于两个假设:弱分代假设(大多数对象短命)和强分代假设(长命对象更长命)。新生代 GC 称为 Minor GC:Eden 满时,复制存活对象到 S0(S1 空闲时),交换 S0/S1 角色;对象年龄(GC 次数)累加,超阈值晋升老年代。动态年龄判定允许 Survivor 满时批量晋升同龄对象。\par
老年代 GC 为 Major 或 Full GC,触发于老年代满、空间分配担保失败或显式 System.gc()。Full GC 回收整个堆和方法区,STW 时间长,最该避免。Minor GC 频率高但暂停短(毫秒级),Major GC 中等频率,Full GC 罕见但代价大。\par
\chapter{6. Java 垃圾收集器详解(HotSpot JVM)}
HotSpot JVM 提供多种收集器,按并行/并发分代组合使用。Serial 是单线程收集器,适合客户端模式,新生代用标记-复制,老年代用标记-整理。ParNew 是其多线程版,常与 CMS 搭配。Parallel(PS)注重吞吐,并行标记-复制新生代和标记-整理老年代。\par
CMS(Concurrent Mark Sweep)追求低延迟,并发标记和清除,仅初始标记和重新标记 STW,但易碎片化。G1(Garbage First)将堆分成 Region(2MB),优先回收垃圾最多的 Region,支持分代,低延迟是 JDK 9+ 默认。ZGC 和 Shenandoah 是超低延迟收集器,使用着色指针(Colored Pointers)实现并发整理,STW 亚毫秒级,适用于多 TB 大堆。\par
典型组合中,高吞吐场景选 Parallel + Parallel Old,低延迟用 G1 或 ParNew + CMS。以 G1 为例,其 RSet(Remembered Set)记录跨 Region 引用,Humongous 对象(>50\%{} Region)单独处理,避免碎片。\par
\chapter{7. GC 参数调优实战}
调优从设置堆大小开始:-Xms 和 -Xmx 设为相同值避免动态调整,-Xmn 指定新生代大小。新生代与老年代比例用 -XX:NewRatio=2(1:2),Survivor 用 -XX:SurvivorRatio=8(Eden:Survivor=8:1)。G1 参数如 -XX:+UseG1GC -XX:MaxGCPauseMillis=200 控制目标暂停。\par
调优步骤:先用 jstat 监控 GC 频率,用 VisualVM 或 JFR 分析堆转储。常见问题如 Full GC 频繁,可增大老年代或调低晋升阈值。以电商高并发场景为例,初始 Full GC 占比 20\%{},调优后 -XX:NewRatio=1 -XX:MaxTenuringThreshold=10,Full GC 降至 1\%{},吞吐提升 30\%{}。原则是遵循分代、避免 Full GC、监控先行。\par
考虑这段调优代码示例:\par
\begin{lstlisting}[language=java]
// JVM 参数:-Xms4g -Xmx4g -Xmn1g -XX:SurvivorRatio=8 -XX:+UseG1GC -XX:MaxGCPauseMillis=100
public class TuningDemo {
    public static void main(String[] args) {
        List<byte[]> list = new ArrayList<>();
        for (int i = 0; i < 100000; i++) {
            list.add(new byte[1024 * 100]);  // 分配 100KB 对象,模拟短命对象
        }
        // 预期:频繁 Minor GC,大对象少量晋升
    }
}
\end{lstlisting}
这段代码分配大量小对象,Eden 快速满触发 Minor GC,G1 高效复制到 Survivor。参数确保新生代占 25\%{},暂停控制在 100ms 内。若无调优,Full GC 会因老年代压力而频发。\par
\chapter{8. GC 日志分析}
开启日志用 -Xlog:gc*:file=gc.log:time,uptime,level,tags。JDK 9+ 格式如 [0.123s][info][gc] GC(0) Pause Young (Normal)(allocation failure),解析显示时间、类型、原因。关键指标包括 GC 时间占比(理想 <5\%{})、吞吐量(>90\%{})和晋升率(高则增大新生代)。\par
用 GCViewer 加载日志,观察曲线:若 Full GC 峰值多,检查晋升失败;STW 陡峭,考虑 G1/ZGC。生产中集成 ELK 栈,实现告警。\par
\chapter{9. 高级主题与未来趋势}
内存泄漏常因集合未清空或线程池泄露,用 jmap -histo 排查。NUMA 优化和大页(HugePage)减少 TLB miss,提升 10-20\%{} 性能。JDK 17+ 的 Epsilon 无 GC,适合短命任务;ZGC 生产就绪,云原生中容器感知 GC(如 -XX:+UseContainerSupport)自适应 CPU/Mem 限制,GraalVM Native Image 则 AOT 编译避开 GC。\par
GC 从判定到收集,核心是平衡回收与暂停。生产实践:预留 25\%{} 堆空间,设置 Full GC 告警,用 JFR 定期诊断。推荐书籍《深入理解 Java 虚拟机》和《Java 性能权威指南》,工具 JMH 测试吞吐、JFR 火焰图分析。\par
常见疑问:G1 何时优于 CMS?答:堆 >4GB 或追求可预测延迟时。欢迎评论区分享你的 GC 调优经验!\par
\chapter{附录}
\textbf{A. 示例代码:引用类型演示}\par
\begin{lstlisting}[language=java]
public class ReferenceDemo {
    public static void main(String[] args) {
        // 软引用示例
        SoftReference<byte[]> softRef = new SoftReference<>(new byte[1024 * 1024 * 100]);  // 100MB
        System.out.println(softRef.get() != null ? "软引用存活" : "已回收");
        // 施压内存,触发 OOM 前软引用回收
        List<byte[]> pressure = new ArrayList<>();
        for (int i = 0; i < 10; i++) pressure.add(new byte[1024 * 1024 * 100]);
    }
}
\end{lstlisting}
这段代码创建 100MB 软引用对象,循环分配更多大数组模拟 OOM 前压力。SoftReference 在内存紧张时 get() 返回 null,证明 GC 回收,避免崩溃。解读:软引用持有器不阻止回收,仅在空间不足时响应,get() 可能返回 null,需检查;适用于缓存,如浏览器图片池。\par
\textbf{B. 参考文献}\par
《深入理解 Java 虚拟机》(周志明)、Oracle JDK 文档、OpenJDK GC 源码。\par
