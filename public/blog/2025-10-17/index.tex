\title{基本的基数排序(Radix Sort)算法}
\author{杨子凡}
\date{Oct 18, 2025}
\maketitle
你是否听说过一种不基于比较的排序算法?它像整理档案一样,逐位、逐层地对数字进行归类,最终神奇地实现有序。本文将带你深入探索基数排序的独特魅力,从核心思想到代码实现,彻底掌握这一高效的非比较排序算法。\par
在排序算法的广阔领域中,主流比较排序算法如快速排序和归并排序的时间复杂度下限为 $O(n\log{n})$,这是基于比较操作的固有限制。然而,基数排序作为一种非比较型整数排序算法,打破了这一瓶颈,其时间复杂度可以达到 $O(k\times{n})$,其中 $k$ 代表数字的最大位数。这种算法在特定场景下,例如处理固定位数的整数数据时,能够展现出超越传统排序方法的效率。本文旨在帮助读者彻底理解基数排序的工作原理,掌握最低位优先(LSD)的实现方式,并深入探讨其优势、局限性以及适用场景,从而为实际应用提供坚实的技术基础。\par
\chapter{基数排序的核心思想}
基数排序的核心思想可以通过一个生动的比喻来理解:想象整理一副扑克牌,我们首先按照花色(如红心、黑桃等)将牌分成几个大类,然后在每个花色内部按照点数(从 A 到 K)进行排序。类似地,基数排序通过逐位处理数字,先按最低位(如个位数)将数字分配到不同的「桶」中,然后按顺序收集,再处理更高位(如十位数、百位数),依次类推,直到最高位处理完毕,最终实现整体有序。这里涉及两个关键概念:基数和关键码。基数指的是每一位数字的取值范围,例如对于十进制数,基数为 10(涵盖 0 到 9);关键码则是排序所依据的每一位数字本身。基数排序主要有两种策略:最低位优先(LSD)和最高位优先(MSD)。LSD 从数字的最低位开始排序,是本文的重点;而 MSD 则从最高位开始,采用递归分治的方式,本文仅作简要提及以供拓展。\par
\chapter{算法步骤详解(以 LSD 为例)}
基数排序的 LSD 实现过程可以分解为几个清晰的步骤,无需依赖图示,我们通过文字描述来构建完整的理解。首先,进行准备工作:确定待排序数组中的最大值 \texttt{max\_{}val},并计算最大数字的位数 $k$,公式为 $k = \lfloor\log _{10}(\text{max\_val})\rfloor+1$,这决定了排序所需的轮数。接下来,从最低位(个位)开始,逐位进行排序。例如,在第一轮中,我们创建 10 个桶,分别对应数字 0 到 9,然后遍历数组,根据每个数字的个位数将其放入对应桶中。完成后,按桶的编号顺序(从 0 到 9)依次将元素收集回原数组。这时,数组在个位数上已经达到局部有序。进入第二轮,我们清空桶,基于十位数重复上述分发和收集过程。关键点在于,必须从低到高排序,因为高位相同的数字,其顺序由低位的排序结果决定,这依赖于排序的稳定性。稳定性确保前一轮的排序成果不被破坏,例如如果两个数字的十位数相同,它们的相对顺序由个位数的排序结果保持。重复这一过程,直到处理完最高位,数组便完全有序。整个算法强调稳定性的重要性,它是基数排序正确性的基石。\par
\chapter{代码实现(以 Python 为例)}
以下是一个完整的 Python 实现基数排序的代码示例,我们将逐段进行详细解读,帮助读者理解每一部分的功能。\par
\begin{lstlisting}[language=python]
def radix_sort(arr):
    # 特殊情况处理:如果数组为空或只有一个元素,直接返回
    if len(arr) <= 1:
        return arr
    
    # 寻找数组中的最大值,以确定最大位数
    max_val = max(arr)
    # 计算最大位数 k,使用对数函数并取整
    exp = 1
    while max_val // exp > 0:
        # 创建 10 个桶,每个桶是一个空列表,用于存放对应数字的元素
        buckets = [[] for _ in range(10)]
        
        # 分发阶段:遍历数组,根据当前位数将元素放入对应桶中
        for num in arr:
            digit = (num // exp) % 10  # 计算当前位数字
            buckets[digit].append(num)  # 将数字放入对应桶
        
        # 收集阶段:按桶顺序(0 到 9)将元素覆盖回原数组
        arr = []
        for bucket in buckets:
            arr.extend(bucket)  # 将每个桶中的元素依次添加到数组
        
        # 更新指数,进入下一位处理
        exp *= 10
    
    return arr
\end{lstlisting}
在这段代码中,我们首先处理边界情况,如果数组长度小于等于 1,则无需排序直接返回。接着,通过 \texttt{max(arr)} 找到最大值 \texttt{max\_{}val},并使用 \texttt{exp} 变量来表示当前处理的位数(初始为 1,代表个位)。主循环继续执行,只要 \texttt{max\_{}val // exp} 大于 0,就意味着还有更高位需要处理。在每一轮循环中,我们创建 10 个桶,用于存放数字 0 到 9 对应的元素。分发阶段通过 \texttt{(num // exp) \%{} 10} 计算每个数字在当前位的值,并将其添加到相应桶中。例如,如果 \texttt{exp} 为 1,则计算个位数;如果 \texttt{exp} 为 10,则计算十位数。收集阶段则按桶的顺序将元素重新组合到数组中,确保稳定性,因为桶内元素顺序保持不变。最后,通过 \texttt{exp *= 10} 更新指数,处理下一位。整个过程重复直到最高位处理完毕,返回排序后的数组。为了验证代码,我们可以使用测试用例,例如输入 \texttt{[170, 45, 75, 90, 2, 802, 24, 66]},并观察每轮排序后的中间结果,例如第一轮后数组在个位数上有序,第二轮后在十位数上局部有序,最终完全有序。\par
\chapter{算法分析}
基数排序的时间复杂度为 $O(k\times{n})$,其中 $k$ 是最大数字的位数,$n$ 是数组长度。详细来说,每一轮分发需要遍历所有 $n$ 个元素,收集同样需要 $O(n)$ 时间,总共进行 $k$ 轮,因此总时间为 $O(k\times{n})$。需要注意的是,当 $k$ 远小于 $n$ 时(例如数字范围较小),基数排序效率很高;但如果数字范围很大(如 $2^{32}$),$k$ 值较大,效率可能不如 $O(n\log{n})$ 的比较排序算法。空间复杂度为 $O(n + r)$,其中 $r$ 是基数(十进制下为 10),因为需要额外的桶空间来存储 $n$ 个元素。稳定性方面,基数排序是稳定的排序算法,这得益于每一轮的分发和收集过程都保持了元素的相对顺序,这对于多关键字排序至关重要。\par
\chapter{进阶讨论与变体}
除了基本的 LSD 实现,基数排序还有多种变体和扩展应用。最高位优先(MSD)基数排序从最高位开始处理,采用递归分治策略,适用于数据分布不均匀的场景,但可能需要额外处理空桶问题。对于负数处理,基本实现无法直接适用,因为负数的位表示不同。一种解决方案是将数组分割为负数和正数两部分,分别进行基数排序:对负数取绝对值排序后再反转顺序,最后与正数部分合并。另一种方法是使用偏移量将所有数转换为非负数后再排序。此外,基数排序可以扩展至其他数据类型,例如字符串或日期。对于字符串,可以将其视为字符序列,按位处理实现字典序排序;对于日期,可以分解为年、月、日等部分,逐部分排序。这些扩展体现了基数排序思想的通用性,关键在于将数据抽象为固定位的关键码序列。\par
回顾基数排序的核心,它通过「按位分配、稳定收集」的方式,实现了高效的非比较排序。其优势包括线性时间复杂度(在 $k$ 较小的情况下)、稳定性以及对整数排序的高效性。然而,基数排序也有局限性:它通常仅适用于整数或具有固定位键的数据类型,需要额外的内存空间,且当 $k$ 值较大时效率可能下降。应用场景广泛,例如电话号码排序、身份证号排序或日期排序,这些场景中关键码由多位组成且位数固定。通过本文的讲解,读者应能掌握基数排序的基本原理和实现,为实际开发提供参考。\par
\chapter{互动与思考}
鼓励读者动手实践,尝试用其他编程语言实现基数排序,或扩展代码以处理负数情况。思考题包括:如果使用不稳定的子排序方法,会发生什么?例如,假设在分发过程中不保持桶内顺序,可能导致前一轮排序结果被破坏,从而无法保证最终有序。另一个思考题是如何修改算法对字符串数组进行字典序排序?这可以通过将字符串视为字符序列,按字符的 ASCII 值逐位处理来实现。通过这些互动,读者可以加深对基数排序的理解,并探索其更多可能性。\par
