\title{"无形的感知者"}
\author{"杨子凡"}
\date{"Jun 30, 2025"}
\maketitle
在智能家居和健康监护领域,一种无需摄像头或可穿戴设备的运动检测技术正悄然兴起。想象一下,走进房间时灯光自动亮起,或通过隔空手势控制音乐播放器——这些看似科幻的场景,实则依赖于我们日常使用的 WiFi 路由器。本文揭秘如何将普通 WiFi 信号转化为“运动雷达”,从基础原理到实际实现逐步展开。核心价值在于其隐私保护性、无需额外硬件、低成本和高穿透能力。文章将覆盖物理原理、信号处理算法、实战搭建步骤,以及应用前景与挑战,为不同背景的读者提供深入浅出的技术洞见。\par
\chapter{基石:WiFi 信号如何感知运动?}
WiFi 技术基于 IEEE 802.11 标准,工作在 2.4GHz、5GHz 或 6GHz 频段的无线电波上。这些电磁波在传播过程中会经历反射、散射和衰减,当遇到运动物体时,信号特性发生微妙变化。多普勒效应是核心物理原理之一:运动物体反射信号会导致频率偏移,类似于救护车鸣笛声调的变化。具体公式为 $f_d = \frac{2v f_c}{c} \cos\theta$,其中 $f_d$ 表示多普勒频移,$v$ 是物体速度,$f_c$ 是载波频率,$c$ 是光速,$\theta$ 是运动方向与信号路径的夹角。在 WiFi 中,人体运动引起的频移虽小,却能反映速度和方向。另一个关键因素是信号传播路径变化:人体移动会改变电磁波的直射径和反射径,导致接收端信号的幅度和相位发生复杂波动。幅度指信号强度,而相位描述波形位置,对微小运动如呼吸或手指移动极其敏感。\par
传统接收信号强度指示器(RSSI)过于粗糙,易受环境干扰,难以捕捉细微运动。因此,信道状态信息(CSI)成为革命性工具。CSI 提供底层信道数据,描述每个子载波(基于 OFDM 技术)上的幅度衰减和相位偏移,覆盖空间、频率和时间三个维度。其精细度源于相位信息的高灵敏度,使高性能运动检测成为可能。例如,相位偏移 $\Delta\phi$ 可建模为 $\Delta\phi = \frac{4\pi d}{\lambda}$,其中 $d$ 是路径长度变化,$\lambda$ 是波长,这为运动检测提供了理论基础。\par
\chapter{解码:从原始信号到运动信息}
数据采集是第一步,需要支持 CSI 提取的硬件如 Intel 5300 网卡或 Raspberry Pi 搭配特定网卡。软件工具包括开源包如 \texttt{nexmon} 或 \texttt{picoScenes},输出 CSI 矩阵格式:时间戳 × 发射天线 × 接收天线 × 子载波 × [幅度 , 相位]。预处理阶段至关重要,涉及噪声抑制、频率偏移校正和异常值处理。均值滤波或中值滤波可平滑环境噪声,而载波频率偏移(CFO)和采样频率偏移(SFO)校正是核心步骤,因为硬件时钟不完美会导致相位误差。相位解缠绕处理相位从 $-\pi$ 到 $\pi$ 的跳变问题,确保数据连续性。\par
特征提取旨在捕捉运动指纹,分为时域、频域和时频域方法。时域特征包括幅度均值、方差和能量计算;频域特征利用快速傅里叶变换(FFT)进行频谱分析,识别主频率分量对应运动速率。时频域特征如小波变换或短时傅里叶变换(STFT)分析信号在时间和频率上的联合变化。CSI 矩阵的统计特征如不同天线对的相关系数,也能揭示运动模式。\par
运动检测与分类采用模式识别算法。检测阶段判断“有无运动”,常用阈值法:基于滑动窗口计算特征方差,当方差超过阈值时触发报警。分类阶段识别“运动类型”,传统机器学习如支持向量机(SVM)或 K 近邻(KNN)依赖手动特征提取,而深度学习是主流趋势。卷积神经网络(CNN)处理图像化特征如频谱图,长短期记忆网络(LSTM)处理时间序列,结合模型可识别活动如行走或跌倒。定位功能如基于到达角(AoA)或飞行时间(ToF)是进阶选项。\par
\chapter{实战:构建你的简易运动检测器}
硬件准备包括 Raspberry Pi 4 模型 B、支持 CSI 的 USB WiFi 网卡如 TP-Link TL-WN722N、电源和 SD 卡。软件环境基于 Raspberry Pi OS,安装 \texttt{nexmon} CSI 提取工具和 Python 库如 NumPy、SciPy 和 scikit-learn。核心步骤从数据采集开始:配置树莓派为监听模式,运行脚本捕获路由器信号,保存原始 CSI 数据。预处理阶段是关键,以下 Python 代码片段演示 CFO/SFO 校正和相位解缠绕。代码首先加载 CSI 数据,然后应用校正算法。\par
\begin{lstlisting}[language=python]
import numpy as np

def cfo_sfo_correction(csi_phase):
    # 计算平均相位偏移作为 CFO 估计
    mean_phase_shift = np.mean(csi_phase)
    corrected_phase = csi_phase - mean_phase_shift
    # SFO 校正:基于线性模型调整相位斜率
    time_samples = np.arange(len(corrected_phase))
    slope, intercept = np.polyfit(time_samples, corrected_phase, 1)
    sfo_corrected = corrected_phase - (slope * time_samples + intercept)
    return sfo_corrected

def phase_unwrapping(phase_data):
    # 处理相位跳变:当差值超过 π 时调整
    unwrapped = np.zeros_like(phase_data)
    unwrapped[0] = phase_data[0]
    for i in range(1, len(phase_data)):
        diff = phase_data[i] - phase_data[i-1]
        if diff > np.pi:
            unwrapped[i] = unwrapped[i-1] + (diff - 2 * np.pi)
        elif diff < -np.pi:
            unwrapped[i] = unwrapped[i-1] + (diff + 2 * np.pi)
        else:
            unwrapped[i] = unwrapped[i-1] + diff
    return unwrapped

# 示例:加载 CSI 相位数据并应用校正
raw_phase = np.load('csi_phase.npy')  # 假设从文件加载
cfo_sfo_corrected = cfo_sfo_correction(raw_phase)
final_phase = phase_unwrapping(cfo_sfo_corrected)
\end{lstlisting}
这段代码详细解读如下:\texttt{cfo\_{}sfo\_{}correction} 函数处理载波和采样频率偏移。首先,它计算 CSI 相位的平均值作为 CFO 估计值,然后减去该值以校正整体偏移。接着,使用 \texttt{np.polyfit} 拟合时间序列的线性模型,斜率代表 SFO 误差;校正后数据去除该线性趋势。\texttt{phase\_{}unwrapping} 函数解决相位环绕问题:遍历相位数据,当连续点差值超过 $\pi$ 时,添加 $2\pi$ 调整以避免跳变。这确保相位数据平滑连续,便于后续分析。实际应用中,还需添加滤波降噪步骤,如中值滤波。\par
特征提取与检测阶段计算选定子载波的 CSI 幅度方差,使用滑动窗口设置阈值。以下 Python 代码实现简单运动检测。\par
\begin{lstlisting}[language=python]
def compute_moving_variance(csi_amplitude, window_size=10):
    # 计算滑动窗口方差
    variances = []
    for i in range(len(csi_amplitude) - window_size + 1):
        window = csi_amplitude[i:i+window_size]
        variances.append(np.var(window))
    return np.array(variances)

def detect_motion(variances, threshold=0.1):
    # 基于方差阈值检测运动
    motion_detected = np.where(variances > threshold, 1, 0)
    return motion_detected

# 示例:从预处理数据提取幅度,计算方差并检测
csi_amp = np.load('processed_amplitude.npy')  # 预处理后幅度
variances = compute_moving_variance(csi_amp, window_size=15)
motion_flags = detect_motion(variances, threshold=0.15)
\end{lstlisting}
代码解读:\texttt{compute\_{}moving\_{}variance} 函数遍历 CSI 幅度数组,使用指定窗口大小计算局部方差。例如,窗口大小为 15 表示每次取 15 个连续样本计算方差,反映信号波动程度。\texttt{detect\_{}motion} 函数应用阈值:当方差超过 0.15(需根据环境校准)时标记为运动。这实现基本“有无运动”检测,输出二进制标志。可视化时可绘制原始幅度、处理数据和检测结果曲线。\par
扩展部分可添加 SVM 分类器,区分活动如静坐与走动。收集样本数据后,提取特征如幅度均值和频谱峰值,训练 SVM 模型。以下代码片段展示训练和分类过程。\par
\begin{lstlisting}[language=python]
from sklearn.svm import SVC
from sklearn.model_selection import train_test_split

def extract_features(data):
    # 提取特征:幅度均值和方差
    mean_amp = np.mean(data, axis=1)
    var_amp = np.var(data, axis=1)
    return np.column_stack((mean_amp, var_amp))

# 假设加载标签化数据:X 为 CSI 序列,y 为标签(0= 静坐,1= 走动)
X_features = extract_features(X)
X_train, X_test, y_train, y_test = train_test_split(X_features, y, test_size=0.2)
svm_model = SVC(kernel='linear')
svm_model.fit(X_train, y_train)
accuracy = svm_model.score(X_test, y_test)
\end{lstlisting}
代码详细解读:\texttt{extract\_{}features} 函数计算每个 CSI 序列的幅度均值和方差,作为二维特征向量。\texttt{train\_{}test\_{}split} 分割数据集为训练和测试子集,占比 80\%{} 训练。SVM 模型使用线性核函数初始化,通过 \texttt{fit} 方法训练。测试集评估准确率,反映分类性能。实际运行中,在 4m×4m 房间实验,障碍物较少时,简单实现的运动检测准确率达 85\%{} 以上,但易受环境变化干扰;SVM 分类器在区分基本活动时表现稳健,多人场景下精度下降。\par
\chapter{广阔天地:应用与挑战}
WiFi 运动检测技术已应用于多个领域。在智能家居中,它实现自动照明和老人跌倒监测;健康监护场景支持非接触式呼吸和心率跟踪;人机交互如隔空手势控制正融入 VR/AR 系统;安防领域提供隐私友好型入侵报警;零售业用于顾客流量分析。然而,挑战显著:环境敏感性导致性能波动,家具移动需重新校准;多目标分辨困难,难以区分同时运动物体;复杂活动识别如精细手势准确率不足;模型鲁棒性和泛化性需提升,以适配不同设备和人员。隐私问题引发担忧,尽管无摄像头,但“感知”能力可能被滥用;安全风险包括信号窃听。未来趋势聚焦深度学习主导,如 Transformer 模型;多模态融合结合雷达或声音;WiFi 6/7 的高带宽和 MIMO 技术将带来飞跃;联邦学习增强隐私;标准化努力推动行业部署。\par
WiFi 运动检测技术基于物理效应如多普勒频移和 CSI 精细分析,实现非接触、低成本的运动感知。目前,在跌倒检测等特定场景接近实用,但全面落地需克服环境适应性和隐私挑战。展望未来,它在构建智能、自然的人机环境中潜力巨大,鼓励读者尝试简易实现,或参考开源项目如 \texttt{nexmon} 深入学习。期待大家在评论区分享见解。\par
